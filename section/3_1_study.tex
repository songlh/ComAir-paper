\subsection{A Taxonomy of Complexity Problems}
\label{sec:tax}



We categorize complexity problems based on their complexity types.
We then study complexity problems in each type from the following three aspects:
1) what are their root causes\footnote{We refer root causes as code constructs 
conducting the inefficient computation, 
following previous works in this area\cite{SongOOPSLA2014,ldoctor}.};
2) how they generate user-perceived slowdown;
3) how they are fixed by developers. 
Table~\ref{tab:study} shows the results. 

\noindent{\underline{\textit{$O(N)$: linear complexity.}}} 
As shown in Table~\ref{tab:study}, 
7 out of \ComBugs studied complexity problems are in linear complexity. 
All of them are caused by a buggy loop, 
whose iteration number scales in terms of input size $N$.

Five of them are characterized by buggy loops that contain serialized I/O operations.
Users could perceive these bugs, 
even though the loop iteration numbers are not large.
Patching these 5 bugs involves aggregating I/O operations 
or completely eliminating unnecessary I/O operations. 
For example, Mozilla\#490742 is caused by bookmarking 
tabs using individual database transactions. 
Even bookmarking 50 tabs can cause a timeout dialog 
window to pop up during a performance failure run. 
To fix this bug, Mozilla developers use one single aggregated transaction 
to bookmark all tabs.
As another example, Mozilla\#344059 is due to saving unchanged 
search engine preferences to SQLite, 
and it is fixed by saving search 
engine preferences
only when some of them are changed.

For the other three bugs,
their buggy loops execute many iterations during performance failure runs.
They are fixed by adding shortcuts to completely skip the buggy loops. 
Taking MySQL\#33948 as an example,
MySQL developers followed a common practice to keep all table entries in the same linked list, 
including both used ones and free ones. 
The buggy loop iterates the linked list and looks for a free entry.
During performance failure runs, 
many table entries are used and the buggy loop must iterate excessively to find a free entry.
To fix this bug, MySQL developers simply separate used entries and free entries,
and keep them in two separated linked lists. 


\noindent{\underline{\textit{$O(N^k)$: polynomial complexity (k>1).}}}
As shown in Table~\ref{tab:study}, 
more than half of the studied complexity bugs are in polynomial complexity. 
Similar to the bugs in linear complexity,
the polynomial complexity is also caused by a buggy loop.
However, {\bf both} of the loop execution number 
and the average loop iteration number
scale as input size $N$.

%\vspace{-0.15in}
\begin{figure}%[ht!]
\centering
\scriptsize{
\subfloat[MySQL\#27287]{\includegraphics[width=0.31\linewidth]{figure/mysql27287-runtime-buggy-fixed-line}\label{fig:mysql27287-time}} 
\subfloat[Apache\#34464]{\includegraphics[width=0.31\linewidth]{figure/apache34464-runtime-buggy-fixed-line}\label{fig:apache34464-time}} 
\subfloat[GCC\#27733]{\includegraphics[width=0.31\linewidth]{figure/gcc27733-runtime-buggy-fixed-line}\label{fig:gcc27733-time}} \\ 
\vspace{-0.1in}
}
\mycaption{fig:time}{How the execution time scales with input size for three complexity problems.}
{We use 10 distinct inputs to plot a curve for each bug.}
%\caption{How the execution time scales with input size for four complexity problems. 
%\footnotesize{(These figures show how the execution time changes with the change of input size for MySQL\#27287, 
% Apache\#34464, Mozilla\#477564, and GCC\#27733. For each complexity problem, we use 10 distinct inputs.)}} 
%\vspace{-0.05in}
%\label{fig:time} 
\end{figure} 

\begin{figure}
\centering
\scriptsize
\lstset{basicstyle=\ttfamily\selectfont,
     morekeywords={+},keepspaces=true,numbers=left}
  \mbox{\lstinputlisting[mathescape,boxpos=t]{figure/apache34464.c}}
  \vspace{-0.05in}
  \mycaption{fig:apache34464}{An Apache bug in $O(N^2)$ complexity and its patch.}
  {The execution cost scales polynomially as the number of chars from \texttt{getchar()}.}
%\caption{An Apache performance problem in $O(N^2)$ complexity and its patch. 
%\footnotesize{(This figure shows the buggy code fragment for Apache\#34464. 
% The execution time scales polynomially with the number of characters read from \texttt{getchar()}.)}}
%\vspace{-0.05in}
%\label{fig:apache34464}
\vspace{-3pt}
\end{figure}

To fix the majority (14/19) of bugs in this category,
developers directly modify the buggy loops, 
whose total iteration numbers scale polynomially.
Take MySQL\#27287 as an illustration.
To fix this bug,
developers add an extra field to save parent \texttt{XML\_NODE}
and completely remove the buggy loop shown in Figure~\ref{fig:mysql27287}.
How the execution time changes after patching for MySQL\#27287 is 
shown in Figure~\ref{fig:mysql27287-time}. 

To fix the other bugs (5/19) in this category,
developers reduce workloads processed by the loops scaling polynomially, 
instead of changing the loops directly.
The buggy code fragment of Apache\#34464 is shown in Figure~\ref{fig:apache34464}.
The \texttt{while} loop at line 5 searches a string \texttt{src}
for a target sub-string \texttt{tgt}.
If the search is unsuccessful, 
a new character returned from \texttt{getchar()} at line 8 will be appended 
to string \texttt{src}, 
and the loop will search \texttt{src} again from the beginning. 
There is an inner loop inside \texttt{indexOf()}, whose total iterations 
scale polynomially in terms of the number of characters returned from \texttt{getchar()}. 
After fixing this bug, the inner loop will only check the most recent \texttt{tgtLen} characters.
The developers do not change the inner loop, 
while reducing the workload it processes.   
How execution time scales before and after patching for 
this bug is shown in Figure~\ref{fig:apache34464-time}.




\noindent{\underline{\textit{$O(e^N)$: exponential complexity.}}}
Four of the studied complexity problems are in exponential complexity. 
These complexity problems are fixed 
either by leveraging memoization to reuse previous results 
or by skipping the computation with exponential complexity for large workloads. 

\begin{figure}
\centering
\scriptsize{
\lstset{basicstyle=\ttfamily\selectfont,
     morekeywords={+},keepspaces=true,numbers=left}
  \mbox{\lstinputlisting[mathescape,boxpos=t]{figure/gcc27733.c}}
  \vspace{-0.05in}
 }
 \mycaption{fig:gcc27733}{A GCC performance problem in exponential complexity.}
 {How many times \texttt{mult\_alg} is invoked scales exponentially
  with the number of 1s in the binary form of input \texttt{t}.}
%\caption{A GCC performance problem in exponential complexity. 
% \footnotesize{(During performance failure runs, how many times \texttt{mult\_alg} is invoked scales exponentially
%  with the number of 1s in the binary form of input \texttt{t}.)}}
%\vspace{-0.1in}
%\label{fig:gcc27733}
\vspace{-2pt}
\end{figure}

Three of the studied performance problems are caused by recursive function calls. 
Taking GCC\#27733 in Figure~\ref{fig:gcc27733} as an example, 
the recursive function \texttt{mult\_alg} computes the best algorithm to multiply \texttt{t}.
In each invocation, \texttt{mult\_alg} will try a set of bitwise 
operations to change input 
\texttt{t} into a smaller number, \texttt{t'}, 
and recursively call itself.
The number of times when \texttt{mult\_alg} is invoked scales exponentially 
in terms of the number of 1s in the binary form of input \texttt{t}.
To optimize this function, 
GCC developers use a hash table, \texttt{alg\_hash}, to record
which \texttt{t} has been processed before and its corresponding result.
However, there is a type declaration error inside the hash table entry,
and this error causes \texttt{t} larger than the maximum unsigned integer to never hit cache.
For large \texttt{t}, \texttt{mult\_alg} is still in exponential complexity. 
After fixing the type declaration error, 
memoization is enabled for large \texttt{t}. 
How the complexity changes after patching for 
GCC\#27733 is shown in Figure~\ref{fig:gcc27733-time}.

GCC\#32540 is in exponential complexity and it is caused by an inefficient loop. 
The loop applies an iterative algorithm to implement a compiler optimization. 
The bug-triggering input contains very complex control and data dependence,  
so that the buggy loop scales exponentially in terms of 
\texttt{if} branches in the input code. 
To fix this bug, developers simply disable the optimization 
once they detect complex control and data dependence.  
 
\subsection{Implications}
\label{sec:study_impli}

We summarize the implications from our study, which can guide
the design and implementation of \Tool.

\noindent{\underline{\textit{Implication 1.}}
To effectively profile algorithmic complexity,
\emph{a set of inputs providing similar code coverage with different input sizes are needed}. 
As we discussed in Section~\ref{sec:compare} (last paragraph),
for majority (25/30) of complexity problems, 
it is fairly easy for developers to get these inputs based 
on users' descriptions
and provided inputs when they report complexity problems.  

\noindent{\underline{\textit{Implication 2.}}}
Our study shows that approximately three-fourths of the studied complexity problems (23/30)
are caused by repeated executions of a loop or a recursive function. 
This finding suggests that algorithmic profiling should \emph{focus 
on code constructs involving loops and recursive functions}.
%These bugs are categorized as polynomial complexity and exponential complexity in Section~\ref{sec:tax}.
Previous work~\cite{SongOOPSLA2014,ldoctor} 
shows that sampling code constructs 
that execute many times in one program run can lower the runtime overhead 
while preserving the same diagnosis or detection capability. 
It is promising to  \emph{apply sampling 
to design production-run algorithmic profiling techniques}. 
 
\noindent{\underline{\textit{Implication 3.}}}
We also examine what types of data structures holding inputs processed 
by buggy loops or recursive functions.
The most two common types of data structures 
are array (11/30) and linked list (9/30).
Other types of data structures are either application-specific or 
can only cover one or two bugs.
This fact inspires us to \emph{design effective algorithms 
for loops (or recursive functions) 
that process an array or a linked list during algorithmic profiling}. 






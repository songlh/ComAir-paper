\subsection{A Taxonomy of Complexity Problems}
\label{sec:tax}

%\vspace{-0.15in}
\begin{figure}%[ht!]
\centering
\scriptsize{
\subfloat[MySQL\#27287]{\includegraphics[width=0.31\linewidth]{figure/mysql27287-runtime-buggy-fixed-line}\label{fig:mysql27287-time}} 
\subfloat[Apache\#34464]{\includegraphics[width=0.31\linewidth]{figure/apache34464-runtime-buggy-fixed-line}\label{fig:apache34464-time}} 
\subfloat[GCC\#27733]{\includegraphics[width=0.31\linewidth]{figure/gcc27733-runtime-buggy-fixed-line}\label{fig:gcc27733-time}} \\ 
\vspace{-0.1in}
}
\mycaption{fig:time}{How the execution time scales with input size for three complexity problems.}
{We use 10 distinct inputs to plot a curve for each bug.}
%\caption{How the execution time scales with input size for four complexity problems. 
%\footnotesize{(These figures show how the execution time changes with the change of input size for MySQL\#27287, 
% Apache\#34464, Mozilla\#477564, and GCC\#27733. For each complexity problem, we use 10 distinct inputs.)}} 
%\vspace{-0.05in}
%\label{fig:time} 
\end{figure} 

We categorize complexity problems based on their complexity types.
We then study complexity problems in each type from the following three aspects:
1) what are their root causes\footnote{We refer root causes as code constructs 
conducting the inefficient computation, 
following previous works in this area~\cite{SongOOPSLA2014,ldoctor}.};
2) how they generate user-perceived slowdown;
3) how they are fixed by developers. 
Table~\ref{tab:study} shows the results. 

\noindent{\underline{\textit{$O(N)$: linear complexity.}}} 
Table~\ref{tab:study} shows that seven complexity bugs are in linear complexity,
and they are caused by a buggy loop, 
whose iteration number scales with input size $N$.
Five of them are characterized by buggy loops with serialized I/O operations.
Users could perceive these bugs, 
even though the loop iteration numbers are not large.
Patching the five bugs involves aggregating I/O operations 
or completely eliminating unnecessary I/O operations. 
For example, Mozilla\#490742 is caused by bookmarking 
tabs using individual database transactions. 
Even bookmarking 50 tabs can cause a timeout dialog 
window to pop up. 
To fix this bug, Mozilla developers use one single aggregated transaction 
to bookmark all tabs.
The other two bugs are caused by a loop with many iterations. 
They are fixed by adding shortcuts to completely skip the buggy loop. 

%As another example, Mozilla\#344059 is due to saving unchanged 
%search engine preferences to SQLite, 
%and it is fixed by saving search 
%engine preferences
%only when some of them are changed.


%Take MySQL\#33948 as an example,
%MySQL developers followed a common practice to keep all table entries in the same linked list, 
%including both used ones and free ones. 
%The buggy loop iterates the linked list and looks for a free entry.
%During performance failure runs, 
%many table entries are used and thus the buggy loop has to 
%iterate excessively to find a free entry.
%To fix this bug, MySQL developers simply separate used entries and free entries 
%into two different linked lists. 



\noindent{\underline{\textit{$O(N^k)$: polynomial complexity ($k<1$).}}}
As shown in Table~\ref{tab:study}, 
more than half of the complexity bugs are in polynomial complexity. 
Similar to the bugs in linear complexity,
the polynomial complexity is also caused by a buggy loop.
However, {\bf both} the loop execution number 
and the average loop iteration number
scale as input size $N$.




%\begin{figure}
\centering
\scriptsize
\lstset{basicstyle=\ttfamily\selectfont,
     morekeywords={+},keepspaces=true,numbers=left}
  \mbox{\lstinputlisting[mathescape,boxpos=t]{figure/apache34464.c}}
  \vspace{-0.05in}
  \mycaption{fig:apache34464}{An Apache bug in $O(N^2)$ complexity and its patch.}
  {The execution cost scales polynomially as the number of chars from \texttt{getchar()}.}
%\caption{An Apache performance problem in $O(N^2)$ complexity and its patch. 
%\footnotesize{(This figure shows the buggy code fragment for Apache\#34464. 
% The execution time scales polynomially with the number of characters read from \texttt{getchar()}.)}}
%\vspace{-0.05in}
%\label{fig:apache34464}
\vspace{-3pt}
\end{figure}

To fix the majority (14/19) of the bugs in this category,
developers directly modify the buggy loops, 
whose total iteration numbers scale polynomially.
Take MySQL\#27287 as an illustration,
to fix this bug,
developers add an extra field to save parent \texttt{XML\_NODE}
and completely remove the buggy loop in Figure~\ref{fig:mysql27287}.
How the execution time changes after patching is 
shown in Figure~\ref{fig:mysql27287-time}. 


To fix the other bugs (5/19),
developers reduce workloads processed by the loops scaling polynomially, 
instead of changing the loops directly.
For example, Apache\#34464 is caused by two nested loops, 
where the inner loop searches a source string from its beginning 
for a target string 
and the outer loop appends one character to the source 
string in each iteration. 
The total iterations of the inner loop scale polynomially with 
the final length of the source string.
To fix this bug, the developers change the inner loop to only 
examine the last few characters of the source string and reduce the workload 
processed by the inner loop.  



\begin{figure}
\centering
\scriptsize{
\lstset{basicstyle=\ttfamily\selectfont,
     morekeywords={+},keepspaces=true,numbers=left}
  \mbox{\lstinputlisting[mathescape,boxpos=t]{figure/gcc27733.c}}
  \vspace{-0.05in}
 }
 \mycaption{fig:gcc27733}{A GCC performance problem in exponential complexity.}
 {How many times \texttt{mult\_alg} is invoked scales exponentially
  with the number of 1s in the binary form of input \texttt{t}.}
%\caption{A GCC performance problem in exponential complexity. 
% \footnotesize{(During performance failure runs, how many times \texttt{mult\_alg} is invoked scales exponentially
%  with the number of 1s in the binary form of input \texttt{t}.)}}
%\vspace{-0.1in}
%\label{fig:gcc27733}
\vspace{-2pt}
\end{figure}

\noindent{\underline{\textit{$O(e^N)$: exponential complexity.}}}
Four complexity problems are in exponential complexity. 
They are fixed 
either by reusing previous results 
or by skipping the computation in exponential complexity for large workloads. 
Three of them are caused by recursive functions. 
Figure~\ref{fig:gcc27733} shows one such example. 
Function \texttt{mult\_alg()} computes the best algorithm to multiply \texttt{t}.
In each invocation, \texttt{mult\_alg()} tries a set of bitwise 
operations to change input 
\texttt{t} to a smaller number, \texttt{t'}, and recursively calls itself.
How many time \texttt{mult\_alg()} is called scales exponentially 
with the number of 1s in the binary form of input \texttt{t}.
To optimize this function, a hash table (\texttt{alg\_hash}) is used to record
which  \texttt{t} has been processed before and its corresponding result.
However, a type declaration error in the hash table entry causes 
\texttt{t} larger than the maximum \texttt{unsigned int} to never hit cache.
For large \texttt{t}, \texttt{mult\_alg()} is still in exponential complexity. 
After fixing the type declaration error, memoization is 
enabled for large \texttt{t}, 
as shown in Figure~\ref{fig:gcc27733-time}.

%How the complexity changes after patching for 
%GCC\#27733 is shown in Figure~\ref{fig:gcc27733-time}.


%GCC\#32540 is in exponential complexity and it is caused by an inefficient loop. 
%The loop applies an iterative algorithm to implement a compiler optimization. 
%The bug-triggering input contains very complex control and data dependence,  
%so that the buggy loop scales exponentially in terms of 
%\texttt{if} branches in the input code. 
%To fix this bug, developers simply disable the optimization 
%once they detect complex control and data dependence.  
 
\subsection{Implications}
\label{sec:study_impli}

We summarize the implications from our study, which can guide
the design and implementation of \Tool.

\noindent{\underline{\textit{Implication 1.}}
To effectively profile algorithmic complexity,
\emph{a set of inputs providing similar code coverage with different input sizes are needed}. 
As we discussed in Section~\ref{sec:compare} (last paragraph),
for majority (25/30) of complexity problems, 
it is fairly easy for developers to get these inputs based 
on users' descriptions
and provided inputs when they report complexity problems.  

\noindent{\underline{\textit{Implication 2.}}}
Our study shows that all complexity problems are caused 
either by a loop or a recursive function.
This finding suggests that algorithmic profiling should \emph{focus 
on code constructs involving loops and recursive functions}.
Our study also shows that 
approximately three-fourths of complexity problems (23/30)
are caused by repeated executions of a loop or a recursive function. 
Previous work~\cite{SongOOPSLA2014,ldoctor} 
demonstrates that sampling code constructs 
executing many times in one program run can lower the runtime overhead 
while still preserving the same diagnosis or detection capability. 
It is promising to \emph{apply sampling 
to design production-run algorithmic profiling techniques}.

 
\noindent{\underline{\textit{Implication 3.}}}
We also examine what types of data structures holding inputs processed 
by buggy loops or recursive functions.
The two most common types of data structures 
are array (11/30) and linked list (9/30).
Other types of data structures are either application-specific or 
can only cover one or two bugs.
This fact inspires us to \emph{design effective algorithms 
for loops that process an array or 
a linked list during algorithmic profiling}. 







\subsection{Experimental Results}
\label{sec:results}

All experimental results are shown in Table~\ref{tab:benchmark_info}.
We discuss how these results can answer 
the previous three research questions as follows.

\subsubsection{RQ1: Effectiveness}
As shown by the ranking numbers in Table~\ref{tab:benchmark_info},
\Tool can effectively identify buggy loops and buggy functions 
among all evaluated loops and executed functions.
The buggy loop and the buggy function are ranked as No. 1 for \textbf{all}
benchmarks by the production-run version and the in-house version respectively. 
The numbers of evaluated or executed code constructs are shown as subscripts of 
the two ``ranking'' columns. 
For the production-run version, the evaluated loops range from 3 to 5.
For the in-house version, there are 11 bugs whose executed functions are more than 
100.  GCC\#46401 has 4249 executed functions, 
which is the largest number.  
Our ranking mechanism can effectively identify
root causes among these code constructs.


Our ranking mechanism can indeed save developers' efforts 
and help identify root causes. 
Take GCC\#8805 as an example,
this bug is caused by a nested loop, 
whose inner loop's total iterations are in polynomial complexity (e.g., $O(N^2)$).
To fix this bug, GCC developers significantly reduce 
the workload processed by the inner loop. 
Among the five evaluated loops, three of them are in superlinear complexity, 
and the production-run version successfully ranks the inner loop as No. 1.
There are in total 1373 executed function during buggy runs,
and 87 of them are in superlinear complexity. 
The buggy function containing the outer loop is ranked as No. 1 
by the in-house version. 
After referring the results of \Tool, 
developers can avoid manually inspecting the large number of  
suspicious loops or functions in superlinear complexity. 
%\commentlh{Discuss another case where the root-cause loop does not 
%have the largest loop iteration number. }

\subsubsection{RQ2: Accuracy}
As shown in Table~\ref{tab:benchmark_info} (the ``Cost Function'' column),
the production-run version of \Tool successfully 
reports the correct complexity for the buggy loops of 
all evaluated benchmark. 
We compute the $R^2$ value between observed cost values and 
cost values predicted by the inferred cost function for each bug 
(the ``$R^2$-Func'' column).  
There are only five bugs, whose computed $R^2$ value is less than $0.99$.
The minimum $R^2$ value is 0.92 for Collections\#429-1. 
Previous work considers $R^2$ value larger than 0.92 
as a good fitting~\cite{rsquare-value}.
Our results show that inferred cost functions can well represent 
observed (input size, cost) pairs. 
We also compute $R^2$ to compare the reported input sizes 
and cost values by the two versions of \Tool. 
For 31 out of 38 benchmarks, 
the $R^2$ values are larger than $0.99$ for both input sizes and cost values. 
For the left seven benchmarks, the $R^2$ values are all larger than $0.92$. 
These results show that \emph{the production-run version of \Tool is 
as accurate as the in-house version. 
}

The $R^2$ values computed by comparing input sizes and cost values 
reported by the production-run version with and without 
instrumentation methods (Section~\ref{sec:opt}) are shown in 
Table~\ref{tab:benchmark_info} 
(the ``$R^2$-Input'' and ``$R^2$-Cost'' columns).
For 31 out of 38 benchmarks, $R^2$ are larger than $0.99$ for 
both input sizes and cost values. 
For the other seven bugs, the computed $R^2$ values are not low, 
and all of them are larger than $0.92$, which is considered as a good fitting. 
The minimum $R^2$ value is $0.93$ for GCC\#1687.
Applied instrumentation methods include
sampling and approximation, and these two can potentially influence profiling results.
However, our experimental results demonstrate that \emph{sampling and approximation 
applied in the production-run version of \Tool do not hurt accuracy. 
}

\subsubsection{RQ3: Overhead}

The runtime overhead for the production-run version of \Tool is 
shown in Table~\ref{tab:benchmark_info} (the ``Overhead w/ ins.'' column). 
In general, the performance is good. 
The runtime overhead is constantly 
under 5\% for all evaluated bugs. 
There are night bugs, whose overhead is less than 1\%, 
and 15 bugs, whose overhead is in the range from 1\% to 3\%.  
The largest overhead is 4.82\% for Collections\#434. 
The results imply that \emph{the production-run version of 
\Tool only incurs a negligible runtime overhead, 
and it can be deployed in production environment.}

We also measure the runtime overhead for the production-run 
version without enabling instrumentation methods, 
including sampling, instrumentation-site optimizations
and approximation (the ``Overhead w/o ins.'' column). 
The overhead is significantly larger, compared with when instrumentation methods enabled. 
For 31 bugs, the overhead is larger than 1X. 
For two bugs, the overhead is larger than 10X.
In summary, our designed instrumentation methods can 
indeed lower the runtime overhead. 
The overhead for the in-house version 
of \Tool is also measured 
(the ``Overhead'' column). 
The overhead can be as large as 1252X for 
Mozilla\#416628.
For 32 out of 38 bugs, the overhead is larger than 10X. 
These results indicate that \emph{the 
production-run version of \Tool can generate results as accurate as the in-house version,
but with a significantly lower overhead. }


\subsection{Discussion and Limitations}

The current design and implementation of \Tool 
only consider complexity problems in one thread. 
To profile multi-threaded programs, 
\Tool needs extra synchronizations on all utilized global data structures, 
and these synchronizations may bring more overhead. 
\Tool also needs more advanced input metrics to precisely profile programs 
in the multi-threaded producer-consumer 
model discussed in previous work~\cite{Aprof2}. 

Although the performance of the production-version 
is good on all evaluated benchmarks, 
if a sampled dynamic instance conducts a lot of computation, 
\Tool may encounter a performance corner case and can still incur non-tolerable slowdown.
For example, if a profiling target is a loop and 
one sampled instance of the loop contains a lot of iterations, 
\Tool may cause observable slowdown. 
To handle this problem, the next version of \Tool could 
be enhanced by recording dynamic information of limited number 
of iterations for each sampled loop instance. 




\newpage
\section{Production-Run Algorithmic Profiling}
\label{sec:online}

In this section, we discuss our design and 
implementation for the production-run version of \Tool. 
For production-run usage, profiles are collected from the user side.
It is very important to keep a low runtime overhead, since
end users cannot tolerate any observable performance slowdown.

To achieve the requirement of production-run usage,
our design follows several principles. 

First, \textit{study-guided}. 
In contrast to the in-house version,
we design the production-run version of \Tool
under the guidance of the empirical study in Section~\ref{sec:study}.
We focus on the majority of complexity problems, 
which are caused either by repeated execution of loops ($N^k$)
or recursive functions ($2^N$).
We focus on common data structures, which are array and linked list.

Second, \textit{focused checking}.
Developers could choose a suspicious loop or recursive function to monitor 
to better understand its complexity and processed workload in production runs.
Our tool could be used together with performance diagnosis tools~\cite{SongOOPSLA2014} 
or traditional profilers,
and focus on loops or recursive functions leading 
to user-perceived performance problems.

Third, \textit{sampling}.
Instead of recording all dynamic information, 
we apply sampling and only record part of information. 
We infer information for the whole execution based on collected samples. 


\subsection{Technical Design}
As we discussed in Section~\ref{sec:study}, 
majority of complexity problems are caused 
caused by repeated executions of a loop or a recursive function. 
Previous works show that sampling code constructs that are executing 
multiple 
times can lower runtime overhead, 
while still able to collect enough runtime information~\cite{SongOOPSLA2014,ldoctor}. 
Inspired by our study and the earlier work, 
we apply sampling to efficiently profile loops and recursive functions with multiple executions. 

As we discussed in Section~\ref{sec:inhouse},
there are, top-down and bottom-up, two methods to analyze RMS+ and DSS collected for multiple dynamic instances for a code construct. 
We leverage the top-down method for the in-house version. 
However, the top-down method is not suitable for sampling. 
The reason is explained as follows. 
For a code construct, \texttt{c}, inside a loop \texttt{l}.
The dynamic instances for \texttt{l} are much less, 
compared with dynamic instances of \texttt{c}.
If we apply the top-down method, 
we need to sample dynamic instances of \texttt{l}, 
and it is very likely that we will miss these instances. 
For each dynamic instance of \texttt{l}, it will conduct a lot of computation.
There will be a very large runtime overhead during collecting dynamic instance of \texttt{l}.
Therefore, we sample instances of \texttt{c}, and infer information for all instances of \textt{c} by using the bottom-up method. 


As we discussed in Section~\ref{sec:inhouse}, 
to infer the complexity function, 
we need select metrics for \textit{cost} and input \textit{size}.

{\underline{\textit{Cost Metric.}}
To efficiently profile a loop, we use the total loop iteration number 
to estimate the total \textit{cost} of all executions.
We initialize a local counter to be 0 at the beginning of the program.
We increase the counter by 1 in the loop header of the monitored loop.
We dump the counter value to log before the program exits. 
Compared with using executed BBs discussed in Section~\ref{sec:inhouse}, 
collecting loop iteration number will incur a smaller overhead.   


{\underline{\textit{Input Size Metric.}}
We still use RMS as input \textit{size} metric for a monitored loop, 
but we will infer RMS from sampled information, 
which is different from what we did in Section~\ref{sec:inhouse}.
The algorithm discussed in Section~\ref{sec:inhouse} 
focuses on memory reads conducted by a single execution 
independently from other executions 
of the same code construct. 
For example, one single execution of 
the buggy loop of MySQL\#27287 (Figure~\ref{fig:mysql27287})
is always linear, as shown in Figure~\ref{fig:mysql27287-in}.
After merging information of different executions from the outer loop,
we can observe the polynomial complexity 
in Figure~\ref{fig:mysql27287-out}.

To estimate RMS for all executions of a monitored loop, 
we first sample memory read and write information 
from several loop executions. 
We then merge sampled information and 
infer RMS for all executions. 

How to sample memory access information is similar to previous 
statistical debugging works~\cite{liblit03,liblit05,CCI,SongOOPSLA2014,ldoctor}.
We make a cloned version of the monitored loop.
We instrument the cloned version to trace memory accesses.
We consider memory accesses conducted by the loop directly
and indirectly. 
Before each execution of the monitored loop, 
we randomly decide whether the instrumented version 
or the original version will execute. 
How many times the instrumented version is executed 
depends on a configurable sampling rate. 
We also add extra marks into the trace to 
differ information sampled from different loop instances. 
We use an off-line analysis to figure out which 
memory addresses are read firstly by a sampled loop instance. 


We apply mark-and-recapture method~\citep{mark-recapture} to estimate RMS. 
Mark-and-recapture is a commonly used statistical method 
to estimate the size of animal population. 
By using this method, a portion of animals is captured, marked, and released. 
Then, another portion is captured.
The size of the whole animal population is estimated 
based on the ratio of marked animals in the second captured portion.  

Since mark-and-recapture needs to compare two captured portions, 
we config our sampling policy to always sample 
two consecutive loop execution instances. 
Assume we have a pair of sampled instances with RMS 
as $N_1$ and $N_2$ respectively.  
If $M$ memory cells read firstly by both of the two loop instances, 
we estimate the RMS for all loop instances as $N=N_1*N_2/M$. 
{\bf xxx I am pretty sure that the current implementation is wrong}







\begin{table}[h!]
\vspace{-0.05in}
  \centering
  \scriptsize
  \newcommand{\Yes}[1]{\checkmark{}$_#1$}
  \newcommand{\No}[0]{-}
  \begin{tabular}{lccccccccc}
    \toprule     
   {\bf BugID}                   &  \multicolumn{4}{c}{Controlled Inputs}            &     \multicolumn{4}{c}{Random Inputs}          & Overhead \\
                           
    \cmidrule(lr){2-5}
    \cmidrule(lr){6-9}
    (\# of runs)                 &  (10)     &   (100)    &    (500)    & (1000)     &  (10)     &   (100)    &    (500)    & (1000)   &   per run\\
    \midrule 

    Mozilla\#347306   & \ding{51}$_{0.96}$ & \ding{51}$_{0.96}$  & \ding{51}$_{0.96}$ & \ding{51}$_{0.96}$ & \ding{51}$_{0.95}$ & \ding{51}$_{0.96}$ & \ding{51}$_{0.96}$ & \ding{51}$_{0.95}$ &  1.31\% \\
    Mozilla\#416628   & \ding{51}$_{0.94}$ & \ding{51}$_{0.95}$  & \ding{51}$_{0.96}$  & \ding{51}$_{0.96}$ & \ding{51}$_{0.88}$ & \ding{51}$_{0.94}$ & \ding{51}$_{0.95}$ & \ding{51}$_{0.95}$ & 0.32\% \\
    Mozilla\#490742   &  -  & -  & - & - & - & - & - & - & 0.33\% \\
    Mozilla\#35294    &  \ding{51}$_{0.99}$  & \ding{51}$_{0.97}$ & \ding{51}$_{0.98}$ & \ding{51}$_{0.98}$ & \ding{51}$_{0.97}$ & \ding{51}$_{0.97}$ & \ding{51}$_{0.98}$ & \ding{51}$_{0.98}$ & 0.94\% \\
    Mozilla\#477564   &  \ding{51}$_{0.98}$  & \ding{51}$_{0.99}$ & \ding{51}$_{0.99}$ & \ding{51}$_{0.99}$ & \ding{51}$_{0.93}$ & \ding{51}$_{0.96}$ & \ding{51}$_{0.97}$ & \ding{51}$_{0.97}$ & 4.55\% \\
    \midrule
    MySQL\#27287      &  \ding{51}$_{0.93}$  & \ding{51}$_{0.93}$ & \ding{51}$_{0.94}$ & \ding{51}$_{0.94}$ & \ding{51}$_{0.86}$ & \ding{51}$_{0.92}$ & \ding{51}$_{0.94}$ & \ding{51}$_{0.94}$& 2.92\% \\
    MySQL\#15811      &  \ding{51}$_{0.93}$  & \ding{51}$_{0.95}$ & \ding{51}$_{0.95}$ & \ding{51}$_{0.96}$ & \ding{51}$_{0.93}$ & \ding{51}$_{0.93}$ & \ding{51}$_{0.94}$ & \ding{51}$_{0.94}$ & 4.09\% \\
    \midrule
    Apache\#37184     &  -  & -  & - & - & - & - & - & - & 4.5\% \\
    Apache\#29743     & \ding{51}$_{0.85}$  & \ding{51}$_{0.85}$ & \ding{51}$_{0.86}$ & \ding{51}$_{0.86}$ & - & - & \ding{51}$_{0.85}$ & \ding{51}$_{0.85}$ & 0.61\% \\
    Apache\#34464     & \ding{51}$_{0.98}$  & \ding{51}$_{0.99}$ & \ding{51}$_{0.99}$ & \ding{51}$_{0.99}$ & \ding{51}$_{0.97}$ & \ding{51}$_{0.99}$ & \ding{51}$_{0.99}$ & \ding{51}$_{0.99}$ & 4.55\% \\
    Apache\#47223     & -  & \ding{51}$_{0.99}$ & \ding{51}$_{0.99}$ & \ding{51}$_{0.99}$ &\ding{51}$_{0.87}$ & \ding{51}$_{0.93}$ & \ding{51}$_{0.94}$ & \ding{51}$_{0.95}$ & 3.7\% \\
    \midrule
    GCC\#46401       & \ding{51}$_{0.89}$  & \ding{51}$_{0.91}$ & \ding{51}$_{0.91}$ & \ding{51}$_{0.92}$ & \ding{51}$_{0.86}$ & \ding{51}$_{0.88}$ & \ding{51}$_{0.91}$ & \ding{51}$_{0.91}$ & 3.5\% \\
    GCC\#1687        & \ding{51}$_{0.85}$  & \ding{51}$_{0.87}$ & \ding{51}$_{0.87}$ & \ding{51}$_{0.87}$ & - & \ding{51}$_{0.86}$ & \ding{51}$_{0.87}$ & \ding{51}$_{0.87}$ & 9.46\% \\
    GCC\#27733       & -  & \ding{51}$_{0.85}$ & \ding{51}$_{0.85}$ & \ding{51}$_{0.85}$ & - & \ding{51}$_{0.86}$ & \ding{51}$_{0.86}$ & \ding{51}$_{0.86}$ & 0.92\% \\
    GCC\#8805        & -  & \ding{51}$_{0.86}$ &\ding{51}$_{0.87}$ &\ding{51}$_{0.87}$ & - & - & \ding{51}$_{0.86}$ &\ding{51}$_{0.87}$ & 1.28\% \\
    GCC\#21430       & \ding{51}$_{0.96}$  & \ding{51}$_{0.96}$ & \ding{51}$_{0.96}$ & \ding{51}$_{0.96}$ & \ding{51}$_{0.85}$ & \ding{51}$_{0.95}$ & \ding{51}$_{0.95}$ & \ding{51}$_{0.95}$ & 3.74\% \\
    GCC\#12322       & \ding{51}$_{0.86}$  &\ding{51}$_{0.87}$ & \ding{51}$_{0.88}$ & \ding{51}$_{0.88}$ & \ding{51}$_{0.85}$ & \ding{51}$_{0.86}$ & \ding{51}$_{0.98}$ & \ding{51}$_{0.88}$ & 0.42\% \\
    \midrule
    \midrule
    Apache\#53622      & \ding{51}$_{0.99}$  & \ding{51}$_{0.98}$ & \ding{51}$_{0.98}$ & \ding{51}$_{0.98}$ & \ding{51}$_{0.94}$ & \ding{51}$_{0.94}$ & \ding{51}$_{0.97}$ & \ding{51}$_{0.97}$ & 3.99\% \\
    Apache\#53637     & \ding{51}$_{0.98}$  & \ding{51}$_{0.99}$ & \ding{51}$_{0.99}$ & \ding{51}$_{0.99}$ & \ding{51}$_{0.96}$ & \ding{51}$_{0.98}$ & \ding{51}$_{0.97}$ & \ding{51}$_{0.98}$ & 4.53\% \\
    Apache\#53803      & \ding{51}$_{0.97}$  & \ding{51}$_{0.97}$ & \ding{51}$_{0.97}$ & \ding{51}$_{0.98}$ & \ding{51}$_{0.97}$ & \ding{51}$_{0.98}$ & \ding{51}$_{0.99}$ & \ding{51}$_{0.98}$ & 1.54\% \\
    Apache\#53821      & \ding{51}$_{0.99}$  & \ding{51}$_{0.97}$ & \ding{51}$_{0.97}$ & \ding{51}$_{0.98}$ & \ding{51}$_{0.90}$ & \ding{51}$_{0.96}$ & \ding{51}$_{0.97}$ & \ding{51}$_{0.97}$ & 2.02\% \\
    Apache\#53822      & \ding{51}$_{0.98}$  & \ding{51}$_{0.98}$ & \ding{51}$_{0.98}$ & \ding{51}$_{0.98}$ & \ding{51}$_{0.92}$ & \ding{51}$_{0.94}$ & \ding{51}$_{0.95}$ & \ding{51}$_{0.96}$ & 4.53\% \\
    \midrule
    Collections\#406    & \ding{51}$_{0.98}$  & \ding{51}$_{0.98}$ & \ding{51}$_{0.98}$ & \ding{51}$_{0.99}$ & \ding{51}$_{0.92}$ & \ding{51}$_{0.98}$ & \ding{51}$_{0.98}$ & \ding{51}$_{0.98}$ & 4.7\% \\
    Collections\#407     & \ding{51}$_{0.98}$  & \ding{51}$_{0.99}$ & \ding{51}$_{0.99}$ & \ding{51}$_{0.99}$ & \ding{51}$_{0.93}$ & \ding{51}$_{0.98}$ & \ding{51}$_{0.98}$ & \ding{51}$_{0.98}$ & 2.69\% \\
    Collections\#408     & \ding{51}$_{0.99}$  & \ding{51}$_{0.97}$ & \ding{51}$_{0.98}$ & \ding{51}$_{0.99}$ & \ding{51}$_{0.95}$ & \ding{51}$_{0.94}$ & \ding{51}$_{0.97}$ & \ding{51}$_{0.98}$ & 2.23\% \\
    Collections\#409     & \ding{51}$_{0.97}$  & \ding{51}$_{0.98}$ & \ding{51}$_{0.98}$ & \ding{51}$_{0.98}$ & \ding{51}$_{0.94}$ & \ding{51}$_{0.95}$ & \ding{51}$_{0.97}$ & \ding{51}$_{0.98}$ & 4.92\% \\
    Collections\#410     & \ding{51}$_{0.98}$  & \ding{51}$_{0.99}$ & \ding{51}$_{0.99}$ & \ding{51}$_{0.99}$ & \ding{51}$_{0.93}$ & \ding{51}$_{0.96}$ & \ding{51}$_{0.97}$ & \ding{51}$_{0.96}$ & 3.92\% \\
    Collections\#412     & \ding{51}$_{0.99}$  & \ding{51}$_{0.97}$ & \ding{51}$_{0.97}$ & \ding{51}$_{0.97}$ & \ding{51}$_{0.97}$ & \ding{51}$_{0.96}$ & \ding{51}$_{0.97}$ & \ding{51}$_{0.97}$ & 2.95\% \\
    Collections\#413    & \ding{51}$_{0.99}$  & \ding{51}$_{0.98}$ & \ding{51}$_{0.99}$ & \ding{51}$_{0.99}$ & \ding{51}$_{0.96}$ & \ding{51}$_{0.96}$ & \ding{51}$_{0.97}$ & \ding{51}$_{0.96}$ & 4.71\% \\
    Collections\#425     & \ding{51}$_{0.99}$  & \ding{51}$_{0.98}$ & \ding{51}$_{0.99}$ & \ding{51}$_{0.99}$ & \ding{51}$_{0.97}$ & \ding{51}$_{0.99}$ & \ding{51}$_{0.99}$ & \ding{51}$_{0.99}$ & 2.69\% \\
    Collections\#426     & \ding{51}$_{0.99}$  & \ding{51}$_{0.98}$ & \ding{51}$_{0.98}$ & \ding{51}$_{0.99}$ & \ding{51}$_{0.91}$ & \ding{51}$_{0.97}$ & \ding{51}$_{0.97}$ & \ding{51}$_{0.98}$ & 3.05\% \\

    Collections\#427     & \ding{51}$_{0.99}$  & \ding{51}$_{0.97}$ & \ding{51}$_{0.97}$ & \ding{51}$_{0.98}$ & \ding{51}$_{0.92}$ & \ding{51}$_{0.94}$ & \ding{51}$_{0.95}$ & \ding{51}$_{0.96}$ & 3.18\% \\
    Collections\#429-0   & \ding{51}$_{0.89}$  & \ding{51}$_{0.89}$ & \ding{51}$_{0.89}$ & \ding{51}$_{0.89}$ & \ding{51}$_{0.86}$ & \ding{51}$_{0.86}$ &\ding{51}$_{0.87}$ & \ding{51}$_{0.88}$ & 14.14\% \\
    Collections\#429-1    & -  & \ding{51}$_{0.86}$ &\ding{51}$_{0.87}$ &\ding{51}$_{0.87}$ & \ding{51}$_{0.86}$ & \ding{51}$_{0.86}$ & \ding{51}$_{0.86}$ & \ding{51}$_{0.86}$ & 13\% \\
    Collections\#429-2    & \ding{51}$_{0.92}$  & \ding{51}$_{0.94}$ & \ding{51}$_{0.94}$ & \ding{51}$_{0.95}$ & \ding{51}$_{0.93}$ & \ding{51}$_{0.94}$ & \ding{51}$_{0.94}$ & \ding{51}$_{0.94}$ & 10.2\% \\
    Collections\#434    & \ding{51}$_{0.97}$  & \ding{51}$_{0.96}$ & \ding{51}$_{0.96}$ & \ding{51}$_{0.97}$ & \ding{51}$_{0.97}$ & \ding{51}$_{0.96}$ & \ding{51}$_{0.96}$ & \ding{51}$_{0.97}$ & 4.66\% \\
    \midrule
    Groovy\#5739-0       & \ding{51}$_{0.99}$  & \ding{51}$_{0.99}$ & \ding{51}$_{0.99}$ & \ding{51}$_{0.98}$ & \ding{51}$_{0.89}$ & \ding{51}$_{0.94}$ & \ding{51}$_{0.95}$ & \ding{51}$_{0.96}$ & 4.53\% \\
    Groovy\#5739-1      & \ding{51}$_{0.99}$  & \ding{51}$_{0.99}$ & \ding{51}$_{0.99}$ & \ding{51}$_{0.99}$ & \ding{51}$_{0.95}$ & \ding{51}$_{0.97}$ & \ding{51}$_{0.99}$ & \ding{51}$_{0.99}$ & 1.93\% \\
    %\midrule
    %\midrule
    %knapsack         &  -  & -  & - & - & - & - & - & - & 8.09\% \\
    %fib              &  -  & -  & - & - & - & - & - & - & 90.70\% \\
    %parentheses     &  \ding{51}$_{0.91}$  & \ding{51}$_{0.91}$  & \ding{51}$_{0.91}$ & \ding{51}$_{0.91}$ & - & \ding{51}$_{0.91}$ & \ding{51}$_{0.91}$ & \ding{51}$_{0.91}$ & 44.81\% \\
    %nqueens         &  -  & -  & - & - & - & - & - & - & 8.74\% \\
    %graphcol      &  -  & -  & - & - & - & - & - & - & 11.87\% \\
    %uts      &  -  & -  & - & - & - & - & - & - & $<$0.01\% \\
    %binomial   &  -  & -  & - & - & - & - & - & - & 25.86\% \\
    %minmax    &  -  & -  & - & - & - & - & - & - & 3.42\% \\


    \bottomrule
   \end{tabular}
  %\nocaptionrule
  \caption{Runtime overhead and profiling capability. \footnotesize{(This table shows runtime 
  overhead and profiling capability 
measured with the default sampling rate (1 out of 100); 
Here, 10, 100, 500, and 1000 represent the different numbers of program runs used for algorithmic profiling;
both controlled inputs and random inputs are used during the experiments.)} 
}
\vspace{-0.35in}
  \label{tab:overhead}
\end{table}
\begin{table}[h!]
  \centering
  \scriptsize
  \newcommand{\Yes}[1]{\checkmark{}$_#1$}
  \newcommand{\No}[0]{-}
  \begin{tabular}{lcccccccccccc}
    \toprule  
     {\bf BugID} & \multicolumn{4}{c}{ Diagnosis Capability} &\multicolumn{4}{c}{Overhead} & \multicolumn{4}{c}{Avg. \# of samples} \\
                           
    \cmidrule(lr){2-5}
    \cmidrule(lr){6-9}
    \cmidrule(lr){10-13}
    (sampling rate)  &($\frac{1}{10^1}$)&($\frac{1}{10^2}$)&($\frac{1}{10^3}$)& ($\frac{1}{10^4}$)  &($\frac{1}{10^1}$) &($\frac{1}{10^2}$)&($\frac{1}{10^3}$)  & ($\frac{1}{10^4}$)  & ($\frac{1}{10^1}$)  & ($\frac{1}{10^2}$)    & ($\frac{1}{10^3}$)   &  ($\frac{1}{10^4}$) \\
    \midrule 

    Mozilla\#347306      & $\surd_{0.96}$ & $\surd_{0.96}$ & $\surd_{0.93}$  & - & 9.76\%  & 0.41\%  & $<$0.01\% & $<$0.01\% & 1.13*$10^{3}$ & 1.2*$10^{2}$ & 13 & 1 \\
    Mozilla\#416628      & $\surd_{0.92}$  & $\surd_{0.92}$  & $\surd_{0.89}$  & -  & 2.3\%  & 0.32\% & $<$0.01\% & $<$0.01\% & 2.47*$10^{3}$ & 2.59*$10^{2}$& 28 & 2  \\
    Mozilla\#490742      &  -  & -  & -  & - & 0.69\% & 0.33\% & 0.11\%  & 0.03\% &  - & - & - & -   \\
    Mozilla\#35294    & $\surd_{0.99}$ & $\surd_{0.99}$ & $\surd_{0.96}$ & $\surd_{0.88}$ & 37.93\% & 15.11\% & 1.33\% & $<$0.01\% & 2.82*$10^{3}$ & 2.94*$10^{2}$ & 31 & 2 \\
    Mozilla\#477564   & -  & $\surd_{0.98}$ & - & $\surd_{0.91}$ & -   & 8.07\% & 5.87\% & 3.47\% & -&2.28*$10^{3}$&2.36*${10^2}$&24 \\
    \midrule
    MySQL\#27287      &$\surd_{0.94}$ & $\surd_{0.93}$ & - & - & 13.34\% & 2.92\% & 0.84\% & $<$0.01\% & 9.9*${10^2}$ & 2.97*${10^2}$ & 39 & 5 \\
    MySQL\#15811      &$\surd_{0.95}$ & $\surd_{0.95}$ & $\surd_{0.93}$ & - &17.56\% & 5.59\% & 0.66\%& $<$0.01\% & $ 2.68*10^{3}$ & 3.85*${2}$ & 41&2 \\
    \midrule
    Apache\#37184     &  -  & -  & -  & - & - & - & -  & - &  - & - & - & -   \\
    Apache\#29743     & $\surd_{0.86}$  & $\surd_{0.85}$  & - & - & 2.35\% & 0.61\% & $<$0.01\% & $<$0.01\% & 5.75*${10^2}$& 63 & 4 & 1 \\
    Apache\#34464     & $\surd_{0.99}$  & $\surd_{0.99}$ &  $\surd_{0.98}$ & - & 47.37\% & 9.94\% & 8.97\% & 6.04\% & 2.81*${10^3}$ & 2.93*${10^2}$ & 30 & 1 \\
    Apache\#47223     & $\surd_{0.99}$  & $\surd_{0.99}$ & - & - & 39.47\% & 3.96\% & 0.75\% & 0.53\% & 2.81*${10^3}$ & 2.93*${10^2}$ & 30 & 1 \\
    \midrule
    GCC\#46401        & $\surd_{0.89}$ & $\surd_{0.88}$  & - & - & 17.66\% & 3.5\% & 0.28\% & $<$0.01\% & 2.35*${10^3}$ & 2.38*${10^2}$ & 25 & 1 \\
    GCC\#1687         & $\surd_{0.87}$ & $\surd_{0.86}$ & - & - & 45.1\% & 11.21\% & 5.19\% & 2.85\% & 6.39*${10^6}$ & 7.12*${10^5}$ & 7.18*$10^{4}$ & 7.18*$10^{3}$ \\
    GCC\#27733        & $\surd_{0.86}$ & $\surd_{0.86}$ & - & - & 25.6\% & 7.95\% & 3.64\% & 0.87\% & 9.5*${10^5}$ & 1.06*${10^5}$ & 1.05*$10^{4}$ & 1.05*$10^{3}$ \\
    GCC\#8805         & $\surd_{0.87}$ & $\surd_{0.86}$ & - & - & 10.35\% & 1.28\% & 0.02\% & $<$0.01\% & 7.78*${10^5}$ & 1.06*${10^5}$ & 1.05*$10^{4}$ & 1.05*$10^{3}$ \\
    GCC\#21430        & $\surd_{0.96}$ & $\surd_{0.96}$ & $\surd_{0.92}$ & - & 19.21\% & 3.74\% & 0.48\% & $<$0.01\% & 5.07*${10^3}$ & 5.26*${10^2}$ & 58 & 3 \\
    GCC\#12322        & $\surd_{0.88}$ & $\surd_{0.87}$ & $\surd_{0.85}$ & - & 4.48\% & 0.42\% & $<$0.01\% & $<$0.01\% & 3.63*${10^4}$ & 3.66*${10^3}$ & 3.75*${10^2}$ & 40 \\
    \midrule
    \midrule
    Apache\#53622     & $\surd_{0.99}$ & $\surd_{0.98}$ & $\surd_{0.98}$ & - & 30.81\% & 3.99\% & 1.82\% & 1.16\% & 6.14*${10^3}$ & 5.9*${10^2}$ & 63 & 5 \\
    Apache\#53637     & $\surd_{0.99}$ & $\surd_{0.98}$ & $\surd_{0.92}$ & - & 49.64\% & 5.19\% & 3.76\% & 2.33\% & 3.02*${10^3}$ & 2.94*${10^2}$ & 31 & 2 \\
    Apache\#53803     & $\surd_{0.99}$ & $\surd_{0.97}$ & $\surd_{0.93}$ & - & 27.25\% & 1.54\% & $<$0.01\% & $<$0.01\% & 2.81*${10^3}$ & 2.95*${10^2}$ & 31 & 2 \\
    Apache\#53821     & $\surd_{0.99}$ & $\surd_{0.97}$ & $\surd_{0.98}$ & - & 34.18\% & 2.02\% & $<$0.01\% & $<$0.01\% & 2.81*${10^3}$ & 2.94*${10^2}$ & 31 & 2 \\
    Apache\#53822     & $\surd_{0.98}$ & $\surd_{0.98}$ & $\surd_{0.93}$ & $\surd_{0.91}$ & 36.23\% & 5.68\% & 4.47\% & 2.39\% & 2.81*${10^3}$ & 2.93*${10^2}$ & 31 & 2 \\
    \midrule
    Collections\#406   & $\surd_{0.99}$ & $\surd_{0.99}$ & $\surd_{0.97}$ & $\surd_{0.87}$ & 39.96\% & 4.70\% & 0.8\% & $<$0.01\% & 2.6*${10^3}$ & 2.75*${10^2}$ & 29 & 2 \\
    Collections\#407   & $\surd_{0.99}$ & $\surd_{0.99}$ & $\surd_{0.97}$ & $\surd_{0.85}$ & 26.72\% & 1.53\% & $<$0.01\% & $<$0.01\% & 2.81*${10^3}$ & 2.93*${10^2}$ & 31 & 2 \\
    Collections\#408   & $\surd_{0.99}$ & $\surd_{0.97}$ & $\surd_{0.99}$ & $\surd_{0.87}$ & 40.38\% & 2.23\% & $<$0.01\% & $<$0.01\% & 5.64*${10^3}$ & 5.89*${10^2}$ & 63 & 2 \\
    Collections\#409   & $\surd_{0.99}$ & $\surd_{0.99}$ & $\surd_{0.93}$ & - & 51.45\% & 4.92\% & 0.37\% & $<$0.01\% & 2.81*${10^3}$ & 2.94*${10^2}$ & 31 & 2 \\
    Collections\#410   & $\surd_{0.99}$ & $\surd_{0.99}$ & $\surd_{0.93}$ & - & 37.60\% & 3.92\% & $<$0.01\% & $<$0.01\% & 2.81*${10^3}$ & 2.94*${10^2}$ & 32 & 2 \\
    Collections\#412   & $\surd_{0.99}$ & $\surd_{0.99}$ & $\surd_{0.97}$ & - & 35.51\% & 2.95\% & $<$0.01\% & $<$0.01\% & 2.81*${10^3}$ & 2.94*${10^2}$ & 31 & 2 \\
    Collections\#413   & $\surd_{0.99}$ & $\surd_{0.99}$ & $\surd_{0.97}$ & - & 38.87\% & 9.87\% & 1.35\% & $<$0.01\% & 2.81*${10^3}$ & 2.94*${10^2}$ & 31 & 2 \\
    Collections\#425   & $\surd_{0.99}$ & $\surd_{0.99}$ & $\surd_{0.98}$ & - & 31.25\% & 1.54\% & $<$0.01\% & $<$0.01\% & 2.81*${10^3}$ & 2.94*${10^2}$ & 31 & 2 \\
    Collections\#426   & $\surd_{0.99}$ & $\surd_{0.99}$ & $\surd_{0.98}$ & - & 27.6\% & 2.41\% & $<$0.01\% & $<$0.01\% & 2.81*${10^3}$ & 2.94*${10^2}$ & 31 & 2 \\
    Collections\#427   & $\surd_{0.99}$ & $\surd_{0.99}$ & $\surd_{0.98}$ & $\surd_{0.86}$ & 46.06\% & 2.74\% & $<$0.01\% & $<$0.01\% & 6.02*${10^3}$ & 5.85*${10^2}$ & 63 & 5 \\
    Collections\#429-0 & $\surd_{0.86}$ & $\surd_{0.86}$ & - & - & 90.81\% & 19.76\% & 3.13\% & 1.09\% & 1.61*${10^3}$ & 1.79*${10^2}$ & 20 & 0 \\
    Collections\#429-1 & $\surd_{0.86}$ & $\surd_{0.86}$ & - & - & 37.79\% & 8.84\% & 6.29\% & 5.56\% & 1.61*${10^3}$ & 1.79*${10^2}$ & 20 & 0 \\
    Collections\#429-2 & $\surd_{0.86}$ & $\surd_{0.86}$ & - & - & 47.96\% & 7.09\% & 0.44\% & 0.02\% & 1.61*${10^3}$ & 1.79*${10^2}$ & 20 & 0 \\
    Collections\#434   & $\surd_{0.99}$ & $\surd_{0.99}$ & $\surd_{0.98}$ & - & 34.73\% & 13.35\% & 12.1\% & 10.61\% & 2.81*${10^3}$ & 2.93*${10^2}$ & 30 & 1 \\
    \midrule
    Groovy\#5739-0     & $\surd_{0.98}$ & $\surd_{0.98}$ & $\surd_{0.97}$ & - & 36.02\% & 5.06\% & 2.91\% & 2.37\% & 2.81*${10^3}$ & 2.93*${10^2}$ & 31 & 2 \\
    Groovy\#5739-1     & $\surd_{0.98}$ & $\surd_{0.98}$ & $\surd_{0.96}$ & - & 36.61\% & 4.96\% & 1.83\% & 1.56\% & 2.81*${10^3}$ & 2.93*${10^2}$ & 31 & 2 \\
    \midrule
    \midrule
    knapsack     & $X_{0.82}$ & $X_{0.82}$ & $X_{0.82}$ & $X_{0.82}$ & 55.92\% & 8.09\% & $<$0.01\% & $<$0.01\% & 2.48*${10^8}$ & 2.58*$10^{7}$ & 2.7*$10^{6}$ & 2.96*$10^{5}$ \\
    fib     & - & - & - & - & 150.63\% & 90.70\% & 58.8\% & 52.82\% & 5.99*${10^4}$ & 6.6*${10^3}$ & 6.7*$10^{2}$ & 70 \\
    parentheses  & $\surd_{0.91}$ & $\surd_{0.9}$ & - & - & 67.57\% & 44.81\% & 35.9\% & 30.58\% & 3.4*${10^7}$ & 3.49*${10^6}$ & 3.53*$10^{5}$ & 3.52*$10^{4}$ \\
    nqueens    & - & - & - & - & 37.86\% & 8.74\% & 4\% & 3.01\% & 4.24*${10^6}$ & 4.72*${10^5}$ & 4.77*$10^{4}$ & 4.73*$10^{3}$ \\
    graphcol  & - & - & - & - & 40.81\% & 11.87\% & 6.34\% & 2.76\% & 1.82*${10^6}$ & 1.86*${10^5}$ & 1.81*$10^{4}$ & 1.85*$10^{3}$ \\
    uts     & - & - & - & - & 18.44\% & 3.42\% & 1.01\% & 0.71\% & 3.51*${10^6}$ & 3.91*${10^5}$ & 3.94*${10^4}$ & 3.91*${10^3}$ \\
    binomial & - & - & - & - & 49.94\% & 25.86\% & 17.58\% & 10.74\% & 1.22*${10^7}$ & 1.35*${10^6}$ & 1.36*$10^{5}$ & 1.35*$10^{4}$ \\
    minma*    & - & - & - & - & - & - & - & - & - & - & - & - \\


    \bottomrule
   \end{tabular}
  %\nocaptionrule
  \caption{Run-time overhead and diagnosis capability evaluated with the default sampling rate (1 out of 10000); 10, 100, 500, 1000 represents the different numbers of success/failure runs used for diagnosis.}
  \label{tab:LBR}
\end{table}




\subsection{Discussions}


\section{Understanding Real-World Complexity Problems}
\label{sec:study}

\begin{table*}[tb!]
\begin{adjustwidth}{-.5in}{-.5in}
\small
\centering
{
\begin{tabular}{|lcccccc|}
\hline
                                                               &   Apache  &   Chrome   &  GCC   &    Mozilla   &   MySQL  &  Total\\
\hline
Total \# of complexity bugs                                          &   4       &    3       &   8    &    10        &   5      &   30 \\
\hline
\multicolumn{7}{|c|}{\bf Taxonomy of complexity problems}\\
\multicolumn{1}{|l}{{\bf $O(N)$}: linear complexity}                 &   1       &    0       &   0    &    4         &   3      &   8\\
\multicolumn{1}{|l}{{\bf $O(N^k)$}: polynomial complexity (k>1)}     &   3       &    3       &   5    &    6         &   1      &  18\\
\multicolumn{1}{|l}{{\bf $O(e^N)$}: exponential complexity}          &   0       &    0       &   3    &    0         &   1      &   4\\
\hline
\multicolumn{7}{|c|}{\bf How complexity problems are reported?}\\
\multicolumn{1}{|l}{How to change input size {\bf is} specified}     &  4&2&5&9&5&25\\
\multicolumn{1}{|l}{How to change input size {\bf not} specified}    &  0&1&3&1&0&5\\
\hline
\end{tabular}
}
\end{adjustwidth}
\caption{Categorization for Section~\ref{sec:study}.
(This table shows how complexity problems distribute among different complexity categories 
 and whether or not how to change input size is specified during reporing.)}
\label{tab:study}
\end{table*}


In this section, we report an empirical study on real-world 
complexity problems. Our empirical study is conducted in two steps.

\begin{enumerate}


\item 
We quantitatively compare complexity problems with non-complexity problems
in a public performance bug benchmark suite. 
We want to understand the popularity of complexity problems 
and whether or not they are different from non-complexity 
problems during reporting and diagnosis. 

\item We build a taxonomy for complexity bugs. 
For bugs in each category, 
we study what code constructs conduct the inefficient computation,
how the user-perceived slowdown is generated, 
and how to fix these bugs. 
Our studying results can guide the design of \Tool.

\end{enumerate}



\subsection{Quantitative Comparison}
\label{sec:compare}

We conduct our study in this section based on a public benchmark 
suite of real-world performance bugs~\cite{PerfBug,SongOOPSLA2014,ldoctor}.
The benchmark suite contains 65 user-perceived performance bugs 
collected from  Apache, Chrome, GCC, Mozilla, and MySQL. 
These software suites cover various types of functionalities and are implemented 
in different programming languages, including C/C++, Java, C\#, and JavaScript. 
All the five studied software suites are large and mature, 
with millions of lines of code and long development histories. 
All bugs are perceived and report by real-world users, and all of them 
have large performance impact. 

After carefully reading the bug reports and the associated buggy code fragments,
we clearly identify \ComBugs performance problems 
caused by {\textit{algorithmic complexity}}.
Developers usually fix these bugs by applying optimized algorithms to lower complexity
or to reduce workloads processed by the buggy code fragments. 
We refer to these performance problems as 
{\textit{complexity problems}} in the remainder of this paper.
In total, there are 30 out of 65 user-perceived performance problems 
caused by algorithmic inefficiency. 
Complexity problems are popular and 
they are indeed one of the major reasons to cause software slowdown.


Bugs not belonging to complexity problems are caused by various reasons.
For example, there are several Mozilla bugs related to GUI, 
like drawing transparent figures or refreshing web pages too frequently. 
There are also bugs caused by misusing synchronizations, 
such as using busy wait or conducting I/O in a critical section. 
We do not consider these bugs as complexity problems.
The numbers of complexity bugs and non-complexity bugs are shown in Figure~\ref{tab:study}.


We conduct hypothesis t-testing~\cite{ttest} 
to compare complexity problems and non-complexity 
problems from the following aspects: why they are introduced, 
how they are reported, and how they are diagnosed. 
We use 95\% as our statistical confidence. 


Misunderstanding real-world workloads and misusing APIs
are two major reasons why performance bugs are introduced. 
Among the 30 complexity bugs, 20 bugs are caused 
by misunderstanding workloads
and 8 bugs are caused by misusing APIs. 
The two numbers for non-complexity bugs are 18 and 7 respectively. 
There is no significant difference between complexity 
problems and non-complexity problems. 


Sometimes, users may compare performance using inputs with different 
sizes~\cite{SongOOPSLA2014} or provide a input with repetitive patterns,  
when reporting performance bugs. 
For both of these two cases, 
it is fairly easy to figure out how to change input sizes to observe performance difference. 
There are 25 complexity bugs falling into one of the two cases, 
while for non-complexity problems, the number is only 8.
There is significant difference. 

On average, developers use 162 days to fix a 
complexity problem and use 103 days to fix a non-complexity problem. 
There is no significant difference. 

\begin{table*}[h!]
  \centering
  \scriptsize
  {
  \newcommand{\Yes}[1]{\checkmark{}$_#1$}
  \newcommand{\No}[0]{-}
  %\resizebox{\textwidth}{70mm}
  {
  \begin{tabular}{lccc|ccccccc|cc}
    \toprule
        & \multicolumn{3}{c}{Benchmark Information} & \multicolumn{7}{c}{Production-run Version} & \multicolumn{2}{c}{In-house Version}   \\

    \cmidrule(lr){2-4}
    \cmidrule(lr){5-11}
    \cmidrule(lr){12-13}
    {BugID} & {KLOC} & {P. L.} & {Comp.} & {Ranking} & $R^2$-Input  & $R^2$-Cost & $R^2$-Input    & $R^2$-Input       & Cost     & {Overhead} & {Ranking} & {Overhead}  \\
                &            &             &             &               & \tiny{P-vs.-I} & \tiny{P-vs.-I} & \tiny{100-vs.-1} & \tiny{100-vs.-1}   & Function &            &  & \\
    \midrule

    \bottomrule
   \end{tabular}
   }
   }
  %\nocaptionrule
  \vspace{0.1in}
   \mycaption{tab:benchmark_info}{Benchmark Information and evaluation results for \Tool.}
   {In the ``Benchmark Information'' columns,
   $x^*$: $x$ thousands of lines of code for re-implemented benchmarks;
   Buggy C. C.: buggy code construct;
   L.: Loop; 
   F.: Lunction.
   In the ``In-house Experiments''columns,
   $x_{y}$: the root-cause function is ranked as $x$th in $y$ 
   functions with the same or higher order of complexity as \texttt{main} function;
   Max Overhead: runtime overhead when the whole program is instrumented;
   Min Overhead: runtime overhead when only the root-cause function and all its callees are instrumented.
   In the ``Production-run Experiments'' columns, 
   }
  %\label{tab:benchmark_info}
\vspace{-0.15in}
\end{table*}

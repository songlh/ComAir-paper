\section{In-house Algorithmic Profiling}
\label{sec:in-house}

In this section, we present the design and implementation 
of \Tool under the in-house setting. 
We choose RMS as input metric and BB as cost 
metric for the in-house version of \Tool, 
since RMS is more general than DSS and BB is a balance 
between monitoring accuracy and monitoring efficiency, 
sharing the same design philosophy as Aprof~\cite{Aprof1,Aprof2}.
Aprof is implemented under Valgrind framework~\cite{valgrind} .
We will first discuss how we collect RMS and BB under LLVM 
framework~\cite{llvm} in Section~\ref{sec:inhouse_impl}.
Aprof simply attributes complexity to each executed function, 
without providing any postmortem analysis. 
We will then present how we rank identified super-linear 
functions in Section~\ref{sec:rank}. 


\subsection{Implementation and Optimization}
\label{sec:inhouse_impl}

We briefly overview the algorithm 
designed in previous works~\cite{Aprof1,Aprof2} 
before discussing our two designed optimizations. 

To count RMS for every function call instance, 
three global variables need to be instrumented:
\texttt{count}, representing current timestamp 
and incremented by one after each function call,
\texttt{ts}, a hash table containing the latest accessing timestamp 
for each memory cell,
and \texttt{S}, a shadow stack maintaining all active functions. 
We need to monitor four types of instructions to 
update the three global variables:
\texttt{call}, \texttt{return}, \texttt{read}, and \texttt{write}.
When a function is called, 
we need to increment \texttt{count} and grow \texttt{S}.
When a function returns, 
we will dump a log with a timestamp 
representing when the function is call 
and the function’s RMS.
For both memory read and write, we need to update \texttt{ts} 
using the current value of \texttt{count}. 
For memory read, we also need to refer \texttt{ts} to decide 
whether to increment RMS for all functions live on \texttt{S}.


The first optimization is to reduce the number of 
instrumentation sites for \texttt{read} and \texttt{write} instructions. 
If write on a memory cell happens earlier, 
following read on the same memory cell will not increase RMS.
For both memory read and write, 
we need to update \texttt{ts} using the current 
value of \texttt{count} and \texttt{count} is only incremented after function call. 
Therefore, given two consecutive memory accesses on the same location,
if there is no function calls in between, 
we don't need to instrument the second one. 
We rely on dominance analysis to implement this optimization. 
We focus on stack memory cells that hold scalar variables 
and only have read and write as uses 
(i.e., not having ``address of'' as uses), 
so that alias analysis is avoided. 

The second optimization is to efficiently count executed BBs.
Instead of incrementing a global counter in every BB,
we apply an algorithm, which was originally designed to efficiently 
count edge events through selectively instrumenting counter update sites 
on CFG~\cite{event-counting}.
The algorithm has already been proved to conduct path 
profiling efficiently~\cite{path-profiling,peter-ase}.  
Since the original algorithm is design to count edge events, 
we split each BB into two and label the event number to be one for 
edges connecting a pair of split BBs, 
leaving the event number as zero for other edges. 
After that, applying the algorithm~\cite{event-counting}
can tell us optimized instrumentation sites. 




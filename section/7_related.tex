%\newpage

\section{Related Works}
\label{sec:related}

\noindent\textbf{Empirical Studies on Performance Problems.}
Many characteristics studies are conducted on real-world performance
problems~\cite{PerfBug,SongOOPSLA2014,ldoctor,Zaman2012MSR,Nistor2013MSR,HuangRegression,SmartphoneStudy,junwen-1, junwen-2}.
\citet{PerfBug} collect 110 real-world performance bugs from 
5 representative applications.
Their study tries to understand root causes of performance problems,
how performance problems are introduced, 
how to expose them, and how to fix them.
In their following works, they focus their studies on
user-perceived performance problems~\cite{SongOOPSLA2014}
and inefficient loops~\cite{ldoctor}.
\citet{Zaman2012MSR} conduct quantitative comparison between performance bugs
and non-performance bugs collected from FireFox and Chrome.
\citet{Nistor2013MSR} compare performance bugs with non-performance bugs
from aspects of discovering, reporting and fixing.
\citet{HuangRegression} try to figure out what code changes are 
more likely to introduce performance regressions by
studying 100 performance regressions.
\citet{SmartphoneStudy} find common inefficiency patterns after
inspecting 70 performance problems from smartphone applications.
There are also research work focusing on studying inefficiency 
in web applications~\cite{junwen-1, junwen-2}.
Similar to our work in Section~\ref{sec:study},
these studies have important findings and can guide technical design to combat
real-world performance problems.
However, our work is the first empirical study focusing on complexity problems,
and it provides important supplement to existing studies.


\noindent\textbf{Profilers.}
Traditional profilers are the most widely used tools
during performance optimization and performance debugging.
After collecting runtime information,
profilers associate performance metrics to executed instructions,
functions, or calling context~\cite{oprofile,gprof, CCT}.
Many research works are proposed to improve
accuracy of profilers~\cite{4Profilers, LagHunter, AppInsight}, or
to reduce runtime overhead~\cite{AdaptiveBurst}
and memory overhead of profilers~\cite{HotCallingContext}.
However, traditional profilers can only analyze one single program run,
while failing to connect results from multiple runs and failing to
predict results for inputs not seen before.


Aprof~\cite{Aprof1, Aprof2} and AlgoProf~\cite{AlgoProf} are existing
algorithmic profiling techniques.
As discussed in Section~\ref{sec:back},
these two techniques suffer from three limitations. 
\Tool enhances existing techniques through significantly 
reducing runtime overhead and providing an effective ranking algorithm. 
\Tool can accurately point out root causes 
and can be deployed in production runs. 



\noindent\textbf{Performance Bug Detection.}
There are many performance bug detectors.
They leverage static or dynamic techniques to
identify performance problems matching specific inefficiency
patterns~\cite{yufei-perf,CLARITY,xiao13:context,PerfBug,Alabama,CARAMEL,XuDataStructure,XuBloatPLDI2009,XuBloatPLDI2010,Cachetor,LoopInvariant,falsesharing}.
For some of them, their detected bugs overlap with complexity problems.
For example, as we discussed in Section~\ref{sec:eva},
all detected performance bugs by Toddler are in $O(N^2)$ complexity.
However, static detectors fail to provide any information
about detected bugs' performance impact.
Dynamic detectors usually incur more than $10\times$ runtime overhead,
and can only be applied for in-house testing using testing inputs.
\Tool only incurs very small overhead and can be deployed in the users' side.
Profiling results generated by \Tool can help developers better
understand real-world workloads, 
detect previously unknown complexity problems exposed under real-world workloads, 
and diagnose performance failures caused by complexity problems.



\noindent\textbf{Input Generation for Performance Testing.}
There are many test input generation techniques to expose performance 
problems~\cite{WISE,EventBreak,slowfuzz}.
WISE~\cite{WISE} generates inputs with small sizes firstly,
then uses these inputs to learn how to restrict conditional branches,
and finally generates large inputs to expose worst-case complexity using learned policies.
EventBreak~\cite{EventBreak} generates inputs to identify
event handlers whose reaction time increases as the application is running.
SlowFuzz~\cite{slowfuzz} generates inputs to expose worst-case complexity using
resource-usage-guided evolutionary search.
These input generation techniques are orthogonal to our work,
and their generated inputs can be used to conduct algorithmic profiling.


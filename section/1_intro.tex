\section{Introduction}
\label{sec:intro}

\subsection{Motivation}
\label{sec:motiv}

As a type of software bug, performance problems~\footnote{We will use performance problems and performance bugs exchangeably,
following previous works in this area~\cite{SongOOPSLA2014,ldoctor}.}
are one major source of software's slowness and inefficiency~\cite{PerfBug,perf.fse10,SongOOPSLA2014,ldoctor,Alabama}. 
Performance problems cannot be optimized away by state-of-the-art compiler optimizations, 
and they widely widely exist in production-run software,
annoying end users and causing economic loss in the field~\cite{PerfBug,SongOOPSLA2014,ldoctor}. 
Several highly-publicized failures have already been caused by performance problems,
such as making a website costing millions of dollars useless~\cite{ACA-health}.
Combating performance problems is urgent.


Many performance problems are caused by algorithmic inefficiency,
such as implementing a linear algorithm in an $O(N^2)$ way.
We refer to these performance problems as complexity problems in our paper.
Our empirical study on a representative performance-bug
benchmark set~\cite{PerfBug,SongOOPSLA2014} shows that
nearly half of user-perceived performance problems are complexity problems.
Complexity problems are usually ranked high in developers' priority list.
For example, Mozilla developers will immediately try to fix 
complexity bugs degrading exponentially~\cite{mozilla35294}.
Addressing complexity problems is an important aspect of fighting performance bugs.

Algorithmic profiling~\cite{Aprof1,Aprof2,AlgoProf} collects profiles from multiple
executions of the same program and attributes complexity to different code constructs, such as a loop or a function,
in the form of a \textit{cost} function of \textit{input} size.
Algorithmic profiling can be used to detect previously unknown complexity problems and
pinpoint the root causes for performance failures caused by complexity problems.
Algorithmic profiling is an effective method to combat complexity problems. 


\if 0

\begin{enumerate}

\item \textit{Coverage}. Complexity problems are due to different types of complexity,
and they are also caused by a large variety of root causes.
A code construct may take inputs and
consume computation resources of various types.
A good algorithmic profiling technique must cover a
large proportion of the various complexity types, root causes, types of inputs and costs.


\item \textit{Accuracy}.
Given an analyzed code construct,
algorithmic profiling needs to accurately identify
how cost scales as input size changes.
%It is desired to conduct algorithmic profiling under the context of the whole program's execution.
%Otherwise, missing how the code construct cooperates with other parts will lead to inaccurate results.
It must consider the entire context of the program's execution.
The profiling results will be inaccurate if how the code construct interact with other parts of the program is missing.

\item \textit{Performance}.
A lower runtime overhead can allow developers to run more tests
under a given time budget during in-house testing.
Production-run algorithmic profiling can help developers
understand how their programs scale
under real-world workloads.
To be deployed in production runs,
techniques must not incur any observable slowdown.


\end{enumerate}

\fi

{\color{red} Why we need to conduct algorithmic profiling in production runs?}
It is necessary to conduct algorithmic profiling in production runs, 
for the following two reasons. 
First, whether a complexity problem can 
cause user-perceived performance 
impact or whether a complexity problem is necessary to be fixed 
highly depend on workload. 
A previous empirical study shows that workload keeps changing and 
developers usually do not have a good understanding of 
real-world workload~\citep{PerfBug}. 
Second, exposing a complexity problem needs to 
execute a piece of code many times. 
In-house testing is usually guided by code coverage, 
and it is not effective enough to expose all complexity problems. 


No existing techniques can conduct algorithmic profiling effectively and efficiently. 
Traditional profilers are the tools most widely used to
diagnose performance failures~\cite{gprof,oprofile}.
Traditional profilers can measure only how much time 
is spent in each code construct during one single run,
while failing to connect information from multiple runs
and failing to provide any indication about how the execution time scales.
Therefore, traditional profilers fail in effectiveness.
Recent techniques on algorithmic 
profiling~\cite{Aprof1,Aprof2,AlgoProf} will incur more than $30\times$ runtime overhead.
They cannot be deployed in production runs and fail in efficiency. 



\begin{figure}
\centering
\scriptsize
\lstset{basicstyle=\ttfamily\selectfont,
     morekeywords={+},keepspaces=true,numbers=left}
  \mbox{\lstinputlisting[mathescape,boxpos=t]{figure/mysql27287.c}}
  \vspace{-0.1in}
  \mycaption{fig:mysql27287}
  {A MySQL performance problem in polynomial complexity with the size of \texttt{items}.}{}
%  {During performance failure runs, 
%  the execution time scales polynomially with the size of \texttt{items}.}
%\caption{A MySQL performance problem in polynomial complexity.
%\footnotesize{(This figure shows the buggy code fragment for MySQL\#27287.
%   During performance failure runs,
%   the execution time scales polynomially with the size of \texttt{items}.)}}
%\vspace{-0.05in}
%\label{fig:mysql27287}
%\vspace{-0.15in}
\end{figure}


One MySQL complexity problem is shown in Figure~\ref{fig:mysql27287}.
The loop searches parent \texttt{XML\_NODE} for function parameter \texttt{nitems},
which presents an array index for another \texttt{XML\_NODE}.
All \texttt{XML\_NODE}s are maintained in array \texttt{items}.
The loop searches by iterating array \texttt{items}
backward from the input and looking for the first \texttt{XML\_NODE}
whose level is one less than the input.
This piece of code looks innocent.
However, there is an outer loop not shown in the figure.
The outer loop will keep calling \texttt{xml\_parent\_tag} using
the next sibling of the previous \texttt{XML\_NODE},
which has $O(N^2)$ complexity in terms of the number of children of a parent \texttt{XML\_NODE}.
Developers may think that using an implementation with $O(N^2)$ complexity is fine,
since an \texttt{XML\_NODE} usually does not have too many children.
However, during performance failure runs,
an \texttt{XML\_NODE} contains tens of thousands of children,
and this leads significant showdown perceived by the end users.

To fix this bug, the developers added an extra field to each \texttt{XML\_NODE} to save its parent,
and this field is initialized when an \texttt{XML\_NODE} is created.
After applying this patch, the code shown in Figure~\ref{fig:mysql27287} was completely removed.
However, it took developers around 5 months to figure this
out.\footnote{We count the time from when developers confirmed this is performance bug
to when the patch was submitted.}
Production-run algorithmic profiling 
can point out the complexity of this piece of code is $O(N^2)$, 
which is exactly the root cause, 
and can also provide workload information 
to confirm the necessity to fix this bug.



\subsection{Our Contributions}
\label{sec:con}

In this paper,
we present a toolchain, \Tool, to effectively conduct algorithmic profiling with
good coverage, accuracy and performance.
\Tool has two settings. %for in-house runs and production runs.
Under the in-house setting,
\Tool is launched using existing tests, tests from end users,
or tests generated by test input generation techniques\cite{KLEE,s2e,dart,EventBreak}.
\Tool will automatically merge profiles from multiple runs and
identify code constructs with super-linear complexity.
Under the production-run setting,
instrumented programs will be distributed to end users
and complexity information will be collected with a negligible runtime overhead.
Developers can leverage this information to understand real-world workloads better
and to diagnose performance failures caused by complexity problems.


We build \Tool through the following three steps.

We first conduct an empirical study to better understand
real-world complexity problems.
All of the studied complexity bugs come from a representative
performance-bug benchmark suite\cite{PerfBug,SongOOPSLA2014}.
To the best of our knowledge, our work is the first study focusing on complexity problems.
We categorize complexity problems into different complexity categories.
We study root causes,
how user-perceived performance impact is generated,
and fix strategies for bugs in each category.
We also investigate the reporting and diagnosis process of complexity problems.
Our findings and implications can motivate future research on complexity problems.
They have already guided our selection of design points when building the in-house version of \Tool,
and inspired us to apply sampling to the production-run version of \Tool.

We then systematically investigate different design points during algorithmic profiling
in order to build the in-house version of \Tool.
We explore two types of input metrics,
three types of cost metrics,
one method to fit the cost curve,
and one method to compare different cost curves.
Our experience shows that naive implementations
to collect runtime information
will lead to extremely large overhead.
We design several optimizations to accelerate information collection
and avoid collecting redundant information.
We evaluate different design points
using $38$ complexity problems from two sources.
Our evaluation shows that after choosing the correct design points,
\Tool can accurately attribute complexity to code constructs
that cause the perceived performance failures.

To build the production-run version of \Tool,
we apply software-based sampling to algorithmic profiling.
Given a code construct with multiple dynamic instances in a program run,
we sample some of the dynamic instances and leverage the mark-and-recapture
method\citep{mark-recapture} to infer information for all instances.
Our evaluation shows that
sampling allows us to significantly lower the runtime overhead.
The evaluation also shows that sampling does not hurt the profiling capability
and does not require to profile more program runs,
not increasing profiling latency, since
the majority of complexity problems are caused by
repeated executions of a loop or a recursive function, as shown by our empirical study.



Specifically, we make the following contributions:

\begin{enumerate}

\item We conduct the first empirical study on real-world complexity problems.
We have several important findings and implications, including
1) around three-fourths of the studied complexity problems are
caused by repeated executions of a loop or a recursive function;
2) for most complexity problems,
the users describe how to change the input size to reproduce the scaling problem during reporting;
and 3) complexity problems usually take a longer time to diagnose and fix,
and more effective tool supports are needed.

\item We design and implement the in-house version of \Tool through
thoroughly investigating different design points during algorithmic profiling.
Our experimental results show that \Tool can effectively analyze performance failures
caused by different types of complexity problems and attribute complexity information accurately.

\item We design the production-run version of \Tool by applying
software-based sampling and a statistical estimation method.
Experimental results show that we can lower the runtime overhead to less than
$5\%$ for $34$ out of $38$ benchmarks,
without sacrificing the profiling capability and the profiling latency.

\end{enumerate}

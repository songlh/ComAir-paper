\section{Introduction}
\label{sec:intro}

%\subsection{Motivation}
%\label{sec:motiv}

As a type of software bug, performance problems\footnote{We will use performance problems,
performance issues,  and performance bugs exchangeably,
following previous works in this area~\cite{SongOOPSLA2014,ldoctor}.}
are one major source of software's slowness and inefficiency~\cite{PerfBug,perf.fse10,SongOOPSLA2014,ldoctor,Alabama}. 
%Performance problems cannot be optimized away by state-of-the-art compiler optimizations, 
Performance problems widely exist in production-run software, leading to
poor user experience and  even economic loss in the field~\cite{PerfBug,SongOOPSLA2014,ldoctor}. 
Several highly-publicized failures have already been caused by performance problems,
such as making a website costing millions of dollars useless~\cite{ACA-health}.
Therefore, combating performance problems is urgent.


Many performance bugs are caused by algorithmic inefficiency,
such as implementing a linear algorithm in an $O(N^2)$ way, 
which cannot be optimized away by state-of-the-art 
compiler optimizations. 
We refer to these performance problems as \emph{complexity problems}  in this paper. 
Our empirical study on a representative performance-bug
benchmark set~\cite{PerfBug,SongOOPSLA2014} shows that
nearly half of the user-perceived performance problems that
\emph{occurred in production environment} are complexity problems.
In addition, complexity problems are usually ranked as high-priority bugs in 
software development.  For example, Mozilla developers will immediately try to fix 
complexity bugs degrading exponentially~\cite{mozilla35294}.
%\commentty{Was this bug marked as high-priority? Or, we could say ``it took only ? days/hours
%to respond"? }
%\commentlh{In the cited bug report, one developer asked what is the performance impact 
%for the reported bug. He gave out several options. If any is true, he would fix the bug soon. 
%One option is ``degrades exponentially with the number of lines'' }


Given the above discussion, addressing complexity problems in
software production runs is an important aspect of fighting performance bugs. 
One commonly used method to diagnose complexity problems
is through algorithmic profiling~\cite{Aprof1,Aprof2,AlgoProf},
which collects profiles from multiple
executions of the same program and attributes complexity to different code constructs, 
such as a loop or a function, in the form of a \textit{cost} function of \textit{input} size.
Therefore, algorithmic profiling can potentially be used to detect complexity problems and
pinpoint the root causes for performance failures caused by complexity problems. 


\begin{figure}
\centering
\scriptsize
\lstset{basicstyle=\ttfamily\selectfont,
     morekeywords={+},keepspaces=true,numbers=left}
  \mbox{\lstinputlisting[mathescape,boxpos=t]{figure/mysql27287.c}}
  \vspace{-0.1in}
  \mycaption{fig:mysql27287}
  {A MySQL performance problem in polynomial complexity with the size of \texttt{items}.}{}
%  {During performance failure runs, 
%  the execution time scales polynomially with the size of \texttt{items}.}
%\caption{A MySQL performance problem in polynomial complexity.
%\footnotesize{(This figure shows the buggy code fragment for MySQL\#27287.
%   During performance failure runs,
%   the execution time scales polynomially with the size of \texttt{items}.)}}
%\vspace{-0.05in}
%\label{fig:mysql27287}
%\vspace{-0.15in}
\end{figure}


One MySQL complexity problem is shown in Figure~\ref{fig:mysql27287}.
The loop searches parent \texttt{XML\_NODE} for function parameter \texttt{nitems},
%which presents an array index for another \texttt{XML\_NODE}.
which is used as an array index. 
All \texttt{XML\_NODE}s are maintained in the array \texttt{items}.
The loop iterates through the array \texttt{items}
backwardly and searches for the first \texttt{XML\_NODE} 
with \texttt{level} one less than the \texttt{XML\_NODE} 
indexed by \texttt{nitems} (lines 7--11).
This piece of code looks innocent.
However, there is an outer loop not shown in the figure.
The outer loop will keep calling \texttt{xml\_parent\_tag} using
the next sibling of the previous \texttt{XML\_NODE},
which has $O(N^2)$ complexity in terms of the number of children of a parent \texttt{XML\_NODE}.
Developers may think that using an implementation with $O(N^2)$ complexity is fine,
since an \texttt{XML\_NODE} usually does not have too many children.
However, during performance failure runs,
an \texttt{XML\_NODE} contains tens of thousands of children,
and this leads significant showdown perceived by the end users.
\commentty{To make the example more readable, can we add ``+", ``-" to indicate
where the bug was and how it was fixed? }
\commentlh{The patch is to completely remove this loop and add something when initializing all XML\_Nodes. 
It is too large and hard to understand.}

To fix this bug, the developers added an extra field to each \texttt{XML\_NODE} to save its parent,
and this field is initialized when an \texttt{XML\_NODE} is created.
After applying this patch, the code shown in Figure~\ref{fig:mysql27287} was completely removed.
However, it took developers around 5 months to figure this
out.\footnote{We count the time from when developers confirmed this is performance bug
to when the patch was submitted.}
Production-run algorithmic profiling 
can point out the complexity of this piece of code is $O(N^2)$, 
which is exactly the root cause, 
and can also provide workload information 
to confirm the necessity to fix this bug.


%There has been some research on designing algorithmic profiling 
%to diagnose performance issues~\cite{gprof,oprofile}. 
%However, these techniques
Existing techniques
are not capable of handling production-run complexity problems
efficiently and effectively due to several reasons. 
First, many algorithmic profilers~\cite{Aprof1,Aprof2,AlgoProf} and performance 
testing tools~\cite{Alabama,PerfBlower} are designed for in-house usage and cannot be 
deployed in production environment due to heavy overhead. 
Recent techniques on algorithmic 
profiling~\cite{Aprof1,Aprof2,AlgoProf} can incur more than $30\times$ runtime overhead.
Second, in-house profilers often  provide limited knowledge of real-world workload, which is necessary 
to determine if a complexity problem can cause user-perceived performance 
impact. A previous empirical study shows that workload keeps changing and 
developers usually do not have a good understanding of real-world workload~\citep{PerfBug}.  
Third, existing profilers intended for production usage~\cite{gprof,oprofile,LagHunter,IntroPerf}
can measure only how much time 
is spent in each code construct during one single run,
but fail to synthesize information from multiple runs
or provide any indication about how the execution time scales.


%\noindent\textbf{Contributions.}
%\label{sec:con}
%
The goal of our research is two fold. First, we want to better 
understand  complexity problems, including 
their root causes, how user-perceived performance impact 
in production environment is generated,
their categories, 
and how the bugs in each category are
fixed.  Toward this end, 
we conduct an empirical on a representative
performance-bug benchmark suite \cite{PerfBug,SongOOPSLA2014},
containing 65 user-perceived  performance bugs from five
real-world applications.  We discovered that 
1) around three-fourths of the studied complexity problems are
caused by repeated executions of a loop or a recursive function;
2) for most complexity problems,
the users describe how to change the input size to reproduce the scaling problem during reporting;
and 3) complexity problems usually take a longer time to diagnose and fix,
and more effective tool supports are needed. 
%\commentlh{We need another one to replace 3). 
%Hypothesis testing shows that there is no significant difference.}
To the best of our knowledge, our work is the first study focusing on complexity problems.
%Our findings and implications can motivate future research on complexity problems.
%They have already guided our selection of design points when building the in-house version of \Tool,
%and inspired us to apply sampling to the production-run version of \Tool.



Second, guided by the findings from the empirical
study, we develop \Tool, the first automated tool to effectively and efficiently 
conduct algorithmic profiling under production-run setting.
Given a program under profiling, \Tool first instruments the program
using a light-weighted instrumentation method. 
The key idea is to use software-based sampling to conduct algorithmic profiling.
Given a code construct, like a loop or a function,  
with multiple dynamic instances in a program run,
we sample some of the dynamic instances to record their input and cost information
and leverage the mark-and-recapture
method \citep{mark-recapture} to infer input and cost information for all instances.
\commentty{Infer what information?}
\commentlh{addressed}
\Tool selectively instruments different code constructs and 
creates multiple program versions distributed to end users.
\Tool then collects runtime profiles as different program versions are running
on different user sites under different workloads.
\commentty{Better define what inputs are...}
Next, \Tool synthesizes the profiles and generates a cost function of input size
for each code construct to describe its complexity. 
\commentty{Any novel ideas in this step?}
\commentlh{No}
Finally, \Tool  reports a ranked list of code constructs in terms of their likelihood
of containing performance bugs due to complexity problems. \commentty{We may need to define code
constructs.} 
\commentlh{addressed}



%{\color{red} TODO: discuss the ranking list}


\Tool provides several benefits over existing performance profilers:
1) \Tool supports production-run algorithmic profiling by taking into account
real-world workloads and can thus discover most of the real complexity problems;
2) \Tool incurs negligible runtime overhead while achieving accurate profiling results;
3) \Tool  simplifies the task of detecting performance bugs related to complexity
problems by providing a ranked list of code constructs in terms of their
likelihood to contain complexity problems;
and 4) \commentlh{\Tool provides information about real-world workloads 
processed by each code construct and guides developers to make better priority 
decisions during performance bug fixing and performance optimization.} 
\commentty{what else?} 

We envision that \Tool can be used in at least two scenarios. 
First, \Tool can be deployed on user end to conduct production-run performance
profiling. The results (i.e., a ranked list of code constructs with their cost functions)
are returned to developers, which can help to localize performance 
bugs and understand 
which code constructs have the most performance impact on the system.
Second, \Tool can be configured as an in-house testing tools used by 
developers and testers. After launched with existing testing inputs, 
\Tool can automatically analyzes executed functions and 
reports their complexity information,
which can help developers predict how each function scales. 
\commentty{how to make it as a testing tool.}


\commentlh{
To evaluate \Tool, we use 38 real-world complexity problems from two sources.
These complexity problems cover different types of 
complexity and different types root-cause code constructs. 
Experimental results show that \Tool can successfully identify the 
root cause for each evaluated bug without reporting 
any false positive under both in-house and production-run settings, 
and incur less than 5\% runtime overhead under the production-run setting, 
while can still generate accurate profiling results.  
} 
Compared to ... \commentty{Need to update the this paragraph based 
on the results.}


In summary, we make the following contributions:

\begin{itemize}

\item We conduct the first empirical study on real-world complexity problems.
The study provides several important findings and implications that can
help developers better understand complexity problems and design tools
to handle them. 


\if 0
\item We design and implement the in-house version of \Tool through
thoroughly investigating different design points during algorithmic profiling.
Our experimental results show that \Tool can effectively analyze performance failures
caused by different types of complexity problems and attribute complexity information accurately.

\fi

\item We develop   \Tool, the first performance profiling tool that is intended to be used
in production environment to effectively detect performance
issues caused by different types of complexity problems and attribute the 
problems to specific code constructs with negligible
overhead.  
\Tool can also be configured as an in-house testing tool. 

\item \commentlh{We conduct thorough experiments and demonstrate the effectiveness, accuracy and performance of \Tool. }

%We implement \Tool and conduct an empirical study to demonstrate 
%its effectiveness and efficiency in detecting complexity problems. 



\end{itemize}

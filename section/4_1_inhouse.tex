\subsection{Postmortem Analysis}
\label{sec:rank}


As we discussed in Section~\ref{sec:limit}, 
existing techniques~\cite{Aprof1,Aprof2,AlgoProf} 
simply attribute complexity to each executed function, 
without providing any future analysis. 
Therefore, they fail to identify root-cause functions for 
performance failures caused by complexity problems. 

To address this problem, we design a ranking algorithm 
with the goal to identify root-cause functions. 
Our algorithm follows three intuitions. 
First, root-cause functions must have the same (or higher) 
complexity as \texttt{main} function.
Second, root-cause functions must consume large computation cost.
Third, since all direct and indirect callers of root-cause functions 
consume more cost and have the same complexity, 
we should rank callee higher than its caller, 
to reduce false positives. 

Our algorithm works as follows. 
We first filter out functions with complexity lower than \texttt{main} function.
We then compute caller-callee relationship using static analysis and RMS logs. 
If function \texttt{A} invokes function \texttt{B}, 
we consider there is a partial order between \texttt{A} and \texttt{B},
and \texttt{B} should be ranked higher ($\texttt{B} \leq \texttt{A}$). 
In the end, we calculate a total order for all executed functions based 
on their partial orders and use the total order as our ranking list. 
If two functions are not comparable 
(neither $\texttt{A} \leq \texttt{B}$ nor $\texttt{B} \leq \texttt{A}$), 
we rank the one with larger cost higher. 


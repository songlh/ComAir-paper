
\begin{table}[tb!]
\begin{adjustwidth}{-.5in}{-.5in}
\centering
{
\scriptsize{
\begin{tabular}{|lcccccc|}
\hline
                                                                                  	&   Apache  &   Chrome   &  GCC   &    Mozilla   &   MySQL  &  Total\\
\hline
\multicolumn{7}{|c|}{\bf \# of Non-Complexity and Complexity Bugs (Section~\ref{sec:compare})}\\
\multicolumn{1}{|l}{Non-Complexity Bugs:}                                                   &   12      &    2       &   1    &    9         &  11      &   35 \\
\multicolumn{1}{|l}{Complexity Bugs:}                                                       &   4       &    3       &   8    &    10        &   5      &   30 \\
\hline \hline
\multicolumn{7}{|c|}{\bf Taxonomy of Complexity Bugs (Section~\ref{sec:tax})}\\
\hline
\multicolumn{7}{|l|}{\bf Complexity Type:} \\
\ \	{$O(N)$:}                              					&   1       &    0       &   0    &    4         &   2      &   7\\
\ \ {$O(N^k)$ ($k>1$):}                						&   3       &    3       &   5    &    6         &   2      &  19\\
\ \ {$O(e^N)$:}                       							&   0       &    0       &   3    &    0         &   1      &   4\\
\hline
\multicolumn{7}{|l|}{\bf Fixing Strategies:}\\
\ \ {{Optimize Buggy Code:}}              								&  3        &    3       &   4    &    9         &   5      &  24 \\
\ \ {Skip Workloads:}              									&  1        &    0       &   4    &    1         &   0      &   6\\
\hline
\end{tabular}
}
}
\end{adjustwidth}
\vspace{0.1in}
\mycaption{tab:study}
{Empirical Studies in Section~\ref{sec:study}.}
{}
%\caption{Categorization for Section~\ref{sec:study}.
%\footnotesize{(This table shows how complexity problems distribute among different complexity categories, how developers fix studied complexity problems,
% and whether or not how to change input size is specified during reporting.)}}
%\label{tab:study}
% \vspace{-0.4in}
\end{table}

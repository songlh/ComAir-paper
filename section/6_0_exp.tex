\section{Evaluation of \Tool}
\label{sec:eva}

\subsection{Methodology}

\noindent{\textbf{Implementation and Experimental Platform.}}
We implement \Tool using LLVM-5.0.0~\cite{llvm}. 
All our experiments are conducted on a machine, 
with Intel Xeon(R) 4110 CPU and Linux 3.10.0 kernel.


\noindent{\textbf{Benchmarks.}}
We collect evaluated complexity problems from two sources. 

The first one is our studied complexity problems in Section~\ref{sec:study}.
We successfully reproduce 17 of them. 
Among them, six are extracted from original Java or JS programs, 
and we re-implement them using C/C++.
The reason is that our current implementation of \Tool can only work on C/C++ programs.
One is extracted from a very old version of Mozilla Firefox. 
We fail to build the original software, 
due to some missing dependent libraries. 
Our re-implementation is based on our thorough understanding of these complexity problems.
In our re-implementation, we maintain their original algorithms, 
data structures and caller-callee relationships. 
For other bugs in our study, 
we fail to use them in our experiments, because they depend on special hardware 
or software, which is not available to us. 
We fail to re-implement them, because their data structures are too complex. 

The second source is from Toddler project~\cite{Alabama}. 
Toddler is a dynamic technique that can detect inefficient loops in Java programs.
In total, developers confirmed 21 bugs detected by Toddler. 
All of them are re-implemented in C/C++ 
in following-up work~\cite{ldoctor}.
We use all the 21 re-implemented bugs in our evaluation.  

In total, we evaluate \Tool on 38 benchmarks, 
including 10 real bugs and 28 re-implemented bugs.
All benchmark information is shown in Table~\ref{tab:benchmark_info}. 
Our evaluated real bugs are all large, and seven of them 
are more than 1 million lines of code.
Our evaluated re-implemented bugs are in small to middle sizes, 
ranging from 91 to 1094 lines of code. 
The majority of bugs used in our experiments are in $O(N^2)$ complexity. 
We also two benchmarks in $O(2^N)$ complexity, 
and {\color{red}{XXX}} benchmarks in O(N) complexity.

\noindent{\bf Experimental Settings.}
For each benchmark, we conduct algorithmic profiling on 10 program 
runs with distinct inputs, following the practice 
in previous works~\cite{joy.asplos13,SongOOPSLA2014}.
For all benchmarks used in our experiments, 
their bug reports contain at least one bug-triggering input. 
For 14 bugs, 
the users explicitly explained how to change input sizes 
to reproduce the described scaling problems 
when they reported these bugs. 
We follow the users' descriptions to generate new inputs. 
For the other benchmarks, it is not difficult to figure out 
how to generate new inputs based on the bad inputs in the bug reports,
since the bad inputs contain repetitive patterns. 
For example, when reporting GCC\#46401,
 the user provided a C file, 
containing an expression with thousands of strings as operands. 
We change the number of strings to generate new inputs. 
For each benchmark, we choose the input sizes of the 10 inputs 
to be an arithmetic sequence with the largest input 
as the input from the bug report and the 
smallest input size that is $1/10$ 
of the largest input.

After collecting a set of (input, cost) pairs for a code construct, 
we first use the maximum cost to aggregate pairs with the same input size into one,
because we want to infer the worst-case complexity.
{\color{red} Todo: how do we infer cost function?}

\noindent{\bf Research Questions to Answer.} 
Our experiments are designed to evaluate whether \Tool 
can provide the desired profiling capability, accuracy and performance. 
Specifically, we aim to answer the following three questions:

\begin{itemize}

\item {\bf RQ1.} We apply sampling and other 
approximation policies to design the production-run version of \Tool. 
Will the production-run version still provide the same algorithmic 
profiling results as the in-house version? 

\item {\bf RQ2.} We design a ranking policy. 
Will it be able to rank the root-cause function higher among all suspicious functions?

\item {\bf RQ3.} What is the runtime overhead of the in-house and 
production-run versions of \Tool. Can the the production-run version really be able to deployed in production runs?



\end{itemize}

\begin{table*}[h!]
  \centering
  \scriptsize
  {
  \newcommand{\Yes}[1]{\checkmark{}$_#1$}
  \newcommand{\No}[0]{-}
  %\resizebox{\textwidth}{70mm}
  {
  \begin{tabular}{lcccc|ccc|ccc}
    \toprule
                  &  \multicolumn{4}{c}{Benchmark Information}           & \multicolumn{3}{c}{In-house Experiments}         & \multicolumn{3}{c}{Production-run Experiments}   \\

    \cmidrule(lr){2-5}
    \cmidrule(lr){6-8}
    \cmidrule(lr){9-11}
    {\bf BugID}    & {\bf KLOC} & {\bf P. L.} & {\bf Complexity} & {\bf Buggy C. C.} & {\bf Ranking} & {\bf Max Overhead} & {\bf Min Overhead} & {\bf Input Fitting} & {\bf Cost Fitting}& {\bf Overhead} \\
    \midrule

    Mozilla\#347306       &  $88$       & C     &   $O(N^{2})$   &  Array L.       &  \ding{51}$_{0.96}$  &  \ding{51}$_{0.96}$    &  56{\bf X}  & 0$_{11}$   &  \ding{51}$_{0.97}$  &  23{\bf X}     \\
    Mozilla\#416628       &  $105$      & C     &   $O(N^{2})$   &  Linked List L. &  \ding{51}$_{0.92}$  &  \ding{51}$_{0.92}$    &  48{\bf X}  & 0$_7$   &  \ding{51}$_{0.92}$ &  20{\bf X}   \\
    Mozilla\#490742       &  $0.157^*$  & JS    &   $O(N)$       & A-L   &  \ding{51}$_{0.99}$  &  /                     &  1{\bf X}   &  -   & / & $<$0.01\%               \\
    Mozilla\#35294        &  $0.195^*$  & C++   &   $O(N^{2})$   & A-L   &  \ding{51}$_{0.99}$  & \ding{51}$_{0.98}$     & 49{\bf X}  & - &  \ding{51}$_{0.98}$ &  19{\bf X}   \\
    Mozilla\#477564       &  $0.116^*$  & JS    &   $O(N^{2})$   & LL-L  &  \ding{51}$_{0.99}$  & \ding{51}$_{0.99}$     & 56{\bf X} & - &  \ding{51}$_{0.86}$ & 20{\bf X}       \\
    \midrule
    MySQL\#27287          &  $995$      & C++   & $O(N^{2})$     & A-L   & \ding{51}$_{0.89}$ & \ding{51}$_{0.89}$ & 31{\bf X} & 0$_8$ &  \ding{51}$_{0.86}$  & 11{\bf X}     \\
    MySQL\#15811          &  $1127$     & C++   & $O(N^{2})$     & A-L   & \ding{51}$_{0.92}$ & \ding{51}$_{0.92}$ & 35{\bf X}& -  & \ding{51}$_{0.89}$ & 18{\bf X}   \\
    \midrule
    Apache\#37184                   &  $0.092^*$  & Java  & $O(N)$     & Loop  & \ding{51}$_{0.92}$ &  /        & 31.7\% & -  & /  & $<$0.01\%     \\
    Apache\#29743     &  $0.257^*$  & Java  & $O(N^{2})$ & Loop  & \ding{51}$_{0.94}$ &  \ding{51}$_{0.94}$ & 32{\bf X} & 0$_5$  & - & $<$0.01\%  \\
    Apache\#34464     &  $0.16^*$   & Java  & $O(N^{2})$ & A-L   & \ding{51}$_{0.98}$ & \ding{51}$_{0.98}$  & 45{\bf X} & - & \ding{51}$_{0.90}$  & 9{\bf X}   \\
    Apache\#47223     &  $0.162^*$  & Java  & $O(N^{2})$ & A-L   & \ding{51}$_{0.99}$ & \ding{51}$_{0.98}$  & 69{\bf X} & -  & \ding{51}$_{0.99}$ & 23{\bf X}  \\
    \midrule
    GCC\#46401        &  $5521$  & C  & $O(N^{2})$ & LL-L & \ding{51}$_{0.90}$ & \ding{51}$_{0.90}$ & 86{\bf X} & 0$_{56}$ & \ding{51}$_{0.92}$ & 9{\bf X}  \\
    GCC\#1687         &  $2099$  & C  & $O(2^{N})$ & R.F. & \ding{51}$_{0.93/0.87}$ & \ding{51}$_{0.93/0.87}$  & 78{\bf X}/16{\bf X}  & 0$_{14}$/-  & - & 3{\bf X}  \\
    GCC\#27733        &  $3217$  & C  & $O(2^{N})$ & R.F. & \ding{51}$_{0.91/0.86}$ & \ding{51}$_{0.91/0.86}$  & 67{\bf X}/7{\bf X}  & 0$_{27}$/-  & - & 2{\bf X}  \\
    GCC\#8805         &  $2538$  & C  & $O(N^{2})$ & LL-L & \ding{51}$_{0.86}$ & \ding{51}$_{0.86}$ & 88{\bf X}  & 0$_{30}$  & \ding{51}$_{0.90}$& 19{\bf X}  \\
    GCC\#21430        &  $3844$  & C  & $O(N^{2})$ & LL-L & \ding{51}$_{0.98}$ & \ding{51}$_{0.97}$ & 80{\bf X} & 0$_{26}$    & \ding{51}$_{0.89}$ & 18{\bf X}   \\
    GCC\#12322        &  $2341$  & C  & $O(N^{2})$ & LL-L & \ding{51}$_{0.92}$ & \ding{51}$_{0.92}$ & 84{\bf X}  & 0$_{37}$    & \ding{51}$_{0.92}$ & 16{\bf X}  \\
    \midrule
    \midrule
    Apache\#53622     & $1.094^*$  & Java  & $O(N^{2})$ & A-L & \ding{51}$_{0.99}$ & \ding{51}$_{0.98}$ & 56{\bf X} & - & \ding{51}$_{0.93}$    & 20{\bf X}     \\
    Apache\#53637     & $0.937^*$  & Java  & $O(N^{2})$ & A-L & \ding{51}$_{0.98}$ & \ding{51}$_{0.98}$  & 44{\bf X} & - & \ding{51}$_{0.93}$   & 22{\bf X}  \\
    Apache\#53803     & $0.421^*$  & Java  & $O(N^{2})$ & A-L & \ding{51}$_{0.98}$  & \ding{51}$_{0.98}$  & 82{\bf X} & - & \ding{51}$_{0.93}$    & 20{\bf X}     \\
    Apache\#53821     & $0.417^*$  & Java  & $O(N^{2})$ & A-L & \ding{51}$_{0.98}$ & \ding{51}$_{0.98}$  & 52{\bf X} & - & \ding{51}$_{0.94}$    & 21{\bf X}      \\
    Apache\#53822     & $0.417^*$  & Java  & $O(N^{2})$ & A-L & \ding{51}$_{0.99}$ & \ding{51}$_{0.99}$  & 54{\bf X} & - & \ding{51}$_{0.93}$    & 20{\bf X}      \\
    \midrule
    Collections\#406      & $0.253^*$  & Java & $O(N^{2})$ & A-L & \ding{51}$_{0.99}$ & \ding{51}$_{0.99}$ & 53{\bf X} & - & \ding{51}$_{0.98}$ & 22{\bf X}       \\
    Collections\#407      & $0.766^*$  & Java & $O(N^{2})$ & A-L & \ding{51}$_{0.97}$ & \ding{51}$_{0.97}$ & 49{\bf X} & - & \ding{51}$_{0.97}$ & 20{\bf X}   \\
    Collections\#408      & $0.742^*$  & Java & $O(N^{2})$ & A-L & \ding{51}$_{0.99}$ & \ding{51}$_{0.99}$ & 57{\bf X} & - & \ding{51}$_{0.97}$ & 20{\bf X}    \\
    Collections\#409      & $0.781^*$  & Java & $O(N^{2})$ & A-L & \ding{51}$_{0.99}$ & \ding{51}$_{0.98}$ & 53{\bf X} & - &\ding{51}$_{0.98}$ & 23{\bf X}     \\
    Collections\#410      & $0.769^*$  & Java  & $O(N^{2})$ & A-L & \ding{51}$_{0.99}$ & \ding{51}$_{0.99}$ & 45{\bf X} & - & \ding{51}$_{0.96}$ & 24{\bf X}    \\
    Collections\#412      & $0.284^*$  & Java & $O(N^{2})$ & A-L  & \ding{51}$_{0.99}$  & \ding{51}$_{0.98}$  & 48{\bf X} & - & \ding{51}$_{0.97}$ & 22{\bf X}     \\
    Collections\#413      & $0.923^*$  & Java & $O(N^{2})$ & LL-L & \ding{51}$_{0.98}$ & \ding{51}$_{0.98}$ & 60{\bf X} & - & \ding{51}$_{0.97}$ & 16{\bf X}   \\
    Collections\#425      & $0.758^*$  & Java & $O(N^{2})$ & A-L & \ding{51}$_{0.99}$ & \ding{51}$_{0.99}$  & 42{\bf X} & - & \ding{51}$_{0.95}$ & 21{\bf X}    \\
    Collections\#426      & $0.783^*$  & Java & $O(N^{2})$ & A-L & \ding{51}$_{0.99}$ & \ding{51}$_{0.99}$  & 58{\bf X} & - & \ding{51}$_{0.97}$ & 21{\bf X}   \\
    Collections\#427      & $0.756^*$  & Java  & $O(N^{2})$ & A-L & \ding{51}$_{0.99}$ & \ding{51}$_{0.99}$  & 58{\bf X} & - & \ding{51}$_{0.97}$ & 24{\bf X}  \\
    Collections\#429-0    & $0.668^*$  & Java & $O(N^{2})$ & Loop & \ding{51}$_{0.95}$ & \ding{51}$_{0.95}$ & 35{\bf X} & - & - & 11{\bf X}      \\
    Collections\#429-1    & $0.536^*$  & Java & $O(N^{2})$ & Loop & \ding{51}$_{0.96}$ & \ding{51}$_{0.96}$ & 40{\bf X} & - & - & 14{\bf X}     \\
    Collections\#429-2    & $0.416^*$  & Java  & $O(N^{2})$ & Loop & \ding{51}$_{0.94}$ & \ding{51}$_{0.94}$  & 21{\bf X} & -  & - & $<$0.01\% \\
    Collections\#434      & $0.336^*$  & Java & $O(N^{2})$ & LL-L & \ding{51}$_{0.99}$ & \ding{51}$_{0.99}$ & 56{\bf X} & - & \ding{51}$_{0.97}$ & 25{\bf X}    \\
    \midrule
    Groovy\#5739-0        & $0.745^*$  & Java  & $O(N^{2})$ & LL-L & \ding{51}$_{0.99}$ & \ding{51}$_{0.98}$ & 55{\bf X} & - & \ding{51}$_{0.90}$ & 21{\bf X}  \\
    Groovy\#5739-1        & $0.756^*$  & Java  &$O(N^{2})$  & A-L &  \ding{51}$_{0.99}$ & \ding{51}$_{0.98}$ & 56{\bf X} & - & \ding{51}$_{0.93}$ & 21{\bf X} \\
    \bottomrule
   \end{tabular}
   }
   }
  %\nocaptionrule
   \caption{Evaluation Results for the in-house version of \Tool.
  \footnotesize{(This table shows benchmark information and evaluation results for the in-house version of \Tool.
   In the benchmark-information columns,
   $x^*$: $x$ thousands of lines of code for re-implemented benchmarks;
   Buggy C.C: buggy code construct;
   A-L: array-processing loop;
   LL-L: linked-list-processing loop; and
   R.F.: recursive function.
   In the RMS+ columns,
   main-C: cost function comparison results for main functions;
   Re-C: cost function comparison results for recursion functions;
   /: Mozilla\#490742 and Apache\#37184 do not have recursion functions;
   \# of F.P.: the number of false positives;
   $x_y$: real ($x$) and benign false positives ($y$);
   For GCC\#1687 and GCC\#27733, we also use RI as cost metric;
   and \ding{51}$_{y}$: calculated similarity using Equation~\ref{eq:sim} is $y$, and y is larger than $0.85$.
    In the DSS columns,
    -: the calculated similarity using Equation~\ref{eq:sim} is less than $0.85$.
)}}
  \label{tab:benchmark_info}
\vspace{-0.15in}
\end{table*}
